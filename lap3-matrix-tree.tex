\documentclass{article}

%%%PACKAGE
\usepackage{amsmath,amssymb}
\usepackage{amsthm}
%% \usepackage[pdftex,bookmarks=true]{hyperref}
\usepackage{cite}
%% \usepackage{enumerate}
% \usepackage{url}
% \usepackage{hyperref}
% \usepackage{systeme}
%\usepackage{siunitx}
% \usepackage{multicol}
% \usepackage{systeme}

% for drawing graphs
\usepackage{tikz}
% \tikzset{every picture/.style={thick}}
\tikzset{every node/.style={draw, circle, inner sep = 2pt}}
\usetikzlibrary{arrows}

% for margins
\usepackage[margin=1in]{geometry}

% for font
\usepackage{euler}
\usepackage[OT1]{eulervm}
\renewcommand{\rmdefault}{pplx}

% \setlength{\parindent}{0pt}  %no indenting

% MACROS
\newcommand{\trans}{^\top}
\newcommand{\adj}{^{\rm adj}}
\newcommand{\cof}{^{\rm cof}}
\newcommand{\inp}[2]{\left\langle#1,#2\right\rangle}
\newcommand{\dunion}{\mathbin{\dot\cup}}
\newcommand{\bzero}{\mathbf{0}}
\newcommand{\bone}{\mathbf{1}}
\newcommand{\ba}{\mathbf{a}}
\newcommand{\bb}{\mathbf{b}}
\newcommand{\bp}{\mathbf{p}}
\newcommand{\bq}{\mathbf{q}}
\newcommand{\bx}{\mathbf{x}}
\newcommand{\by}{\mathbf{y}}
\newcommand{\bz}{\mathbf{z}}
\newcommand{\bu}{\mathbf{u}}
\newcommand{\bv}{\mathbf{v}}
\newcommand{\bw}{\mathbf{w}}
\newcommand{\tr}{\operatorname{tr}}
\newcommand{\nul}{\operatorname{null}}
\newcommand{\rank}{\operatorname{rank}}
%\newcommand{\ker}{\operatorname{ker}}
\newcommand{\range}{\operatorname{range}}
\newcommand{\Col}{\operatorname{Col}}
\newcommand{\Row}{\operatorname{Row}}
\newcommand{\spec}{\operatorname{spec}}
\newcommand{\vspan}{\operatorname{span}}
% \newenvironment{sol}{\medskip\noindent {\bf Solution.}}{\newpage}
\newcommand{\mystrut}{\rule[-.5\baselineskip]{0pt}{2\baselineskip}}
% \newcommand{\mul}{\operatorname{mul}}
\newcommand{\even}{\operatorname{even}}
\newcommand{\sgn}{\operatorname{sgn}}
\newcommand{\iner}{\operatorname{iner}}

%%%COMMENT
\usepackage{soul}
\usepackage{cancel}
\newcommand{\rbf}[1]{\textbf{\color{red}#1}}

%%%THEOREM
\newtheorem{theorem}{Theorem}[section]
\newtheorem{lemma}[theorem]{Lemma}
\newtheorem{proposition}[theorem]{Proposition}
\newtheorem{corollary}[theorem]{Corollary}

\theoremstyle{definition}
\newtheorem{definition}[theorem]{Definition}
\newtheorem{observation}[theorem]{Observation}
\newtheorem{remark}[theorem]{Remark}
\newtheorem{example}[theorem]{Example}
\newtheorem{notation}[theorem]{Notation}
\newtheorem{question}[theorem]{Question}

% for title
\title{Matrix-tree theorem}
\date{\vspace{-1cm}}
\begin{document}
\maketitle
\large

Let $G$ be a connected graph on $n$ vertices.  A \emph{spanning tree} of $G$ is a spanning subgraph $T$ of $G$ such that $T$ is a tree.  Let $L$ be the Laplacian matrix of $G$ and $L(1)$ the principal submatrix obtained from $L$ by removing the first row and column.  Then the \textbf{matrix-tree theorem} states that 
\[\det(L(1)) = \text{the number of spanning trees of $G$}.\]

One way to prove the matrix-tree theorem utilizes the \textbf{Cauchy--Binet formula}:  Let $A$ be an $n\times m$ matrix and $B$ an $m\times n$ matrix.  Then 
\[\det(AB) = \sum_{\substack{\alpha\subseteq[m]\\|\alpha| = n}}\det(A[:,\alpha])\det(B[\alpha,:]),\]
where $A[:,\alpha]$ is the submatrix of $A$ induced on the columns in $\alpha$ while $B[\alpha,:]$ is the submatrix of $B$ induced on the rows in $\alpha$.  

Let $G$ be a graph on $n$ vertices and $m$ edges.  Let $N$ be an incidence matrix of $G$ and $N'$ the $(n-1)\times m$ matrix obtained from $N$ by removing the first row.  Thus, for any $\alpha\subseteq E(G)$ with $|\alpha| = n-1$, we have 
\begin{enumerate}
\item $L(1) = N'N'{}\trans$, 
\item $\det(N'[:,\alpha]) = \pm 1$ if $\alpha$ induces a spanning tree, and 
\item $\det(N'[:,\alpha]) = 0$ if $\alpha$ contains a cycle.
\end{enumerate}
Consequently, 
\[\det(L(1)) = \sum_{\substack{\alpha\subseteq E(G)\\|\alpha| = n-1}}\det(N'[:,\alpha])\det(N'[\alpha,:])\]
counts exactly the number of spanning trees.

\section*{Problems}
\begin{enumerate}
\setlength\itemsep{2em}
\item Pick a $2\times 3$ matrix $A$ and a $3\times 2$ matrix $B$.  Check if the Cauchy--Binet formula is true.
\item Pick a connected graph on $4$ vertices.  Pick a set $\alpha$ of $3$ edges such that $\alpha$ induces a spanning tree.  Observe $N'[:,\alpha]$.  Can you permute the rows and the columns to make it an upper triangular matrix?  (Therefore, the determinant is $\pm 1$.)
\item Pick a connected graph on $4$ vertices.  Pick a set $\alpha$ of $3$ edges such that $\alpha$ contains a cycle.  Observe $N'[:,\alpha]$.  Can you find a vector $\bu$ with entries in $\{1,-1,0\}$ such that $N'[:,\alpha]\bu = \bzero$?  (Therefore, the determinant is $0$.)
\item The number of spanning trees of a complete graph $K_n$ is also the number of labelled trees on $n$ vertices.  How many spanning trees does $K_n$ have?  (This number is called the \textbf{Cayley formula}.)
\item If $G$ has no spanning tree, then $G$ is disconnected.  According to the matrix-tree theorem, $\det(L(1)) = 0$.  Prove, without using the matrix-tree theorem, that $\det(L(1)) = 0$ if the graph is disconnected.
\end{enumerate}

% \newpage
% \section*{Questions to ponder}
% \begin{enumerate}
% \item 
% \end{enumerate}

\end{document}
