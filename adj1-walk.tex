\documentclass{article}

%%%PACKAGE
\usepackage{amsmath,amssymb}
\usepackage{amsthm}
%% \usepackage[pdftex,bookmarks=true]{hyperref}
\usepackage{cite}
%% \usepackage{enumerate}
% \usepackage{url}
% \usepackage{hyperref}
% \usepackage{systeme}
%\usepackage{siunitx}
% \usepackage{multicol}
% \usepackage{systeme}

% for drawing graphs
\usepackage{tikz}
% \tikzset{every picture/.style={thick}}
\tikzset{every node/.style={draw, circle, inner sep = 2pt}}
\usetikzlibrary{arrows}

% for margins
\usepackage[margin=1in]{geometry}

% for font
\usepackage{euler}
\usepackage[OT1]{eulervm}
\renewcommand{\rmdefault}{pplx}

% \setlength{\parindent}{0pt}  %no indenting

% MACROS
\newcommand{\trans}{^\top}
\newcommand{\adj}{^{\rm adj}}
\newcommand{\cof}{^{\rm cof}}
\newcommand{\inp}[2]{\left\langle#1,#2\right\rangle}
\newcommand{\dunion}{\mathbin{\dot\cup}}
\newcommand{\bzero}{\mathbf{0}}
\newcommand{\bone}{\mathbf{1}}
\newcommand{\ba}{\mathbf{a}}
\newcommand{\bb}{\mathbf{b}}
\newcommand{\bp}{\mathbf{p}}
\newcommand{\bq}{\mathbf{q}}
\newcommand{\bx}{\mathbf{x}}
\newcommand{\by}{\mathbf{y}}
\newcommand{\bz}{\mathbf{z}}
\newcommand{\bu}{\mathbf{u}}
\newcommand{\bv}{\mathbf{v}}
\newcommand{\bw}{\mathbf{w}}
\newcommand{\tr}{\operatorname{tr}}
\newcommand{\nul}{\operatorname{null}}
\newcommand{\rank}{\operatorname{rank}}
%\newcommand{\ker}{\operatorname{ker}}
\newcommand{\range}{\operatorname{range}}
\newcommand{\Col}{\operatorname{Col}}
\newcommand{\Row}{\operatorname{Row}}
\newcommand{\spec}{\operatorname{spec}}
\newcommand{\vspan}{\operatorname{span}}
% \newenvironment{sol}{\medskip\noindent {\bf Solution.}}{\newpage}
\newcommand{\mystrut}{\rule[-.5\baselineskip]{0pt}{2\baselineskip}}
% \newcommand{\mul}{\operatorname{mul}}
\newcommand{\even}{\operatorname{even}}
\newcommand{\sgn}{\operatorname{sgn}}
\newcommand{\iner}{\operatorname{iner}}

%%%COMMENT
\usepackage{soul}
\usepackage{cancel}
\newcommand{\rbf}[1]{\textbf{\color{red}#1}}

%%%THEOREM
\newtheorem{theorem}{Theorem}[section]
\newtheorem{lemma}[theorem]{Lemma}
\newtheorem{proposition}[theorem]{Proposition}
\newtheorem{corollary}[theorem]{Corollary}

\theoremstyle{definition}
\newtheorem{definition}[theorem]{Definition}
\newtheorem{observation}[theorem]{Observation}
\newtheorem{remark}[theorem]{Remark}
\newtheorem{example}[theorem]{Example}
\newtheorem{notation}[theorem]{Notation}
\newtheorem{question}[theorem]{Question}

% for title
\title{Walk}
\date{\vspace{-1cm}}
\begin{document}
\maketitle
\large

Let $G$ be a simple graph.  The \emph{adjacency matrix} of $G$ is $A(G) = \begin{bmatrix} a_{ij} \end{bmatrix}$ such that $a_{ij} = 1$ if and only if $\{i,j\}\in E(G)$.  Naturally, the digraph $\Gamma(G)$ of $G$ is a weighted digraph obtained from $G$ by replacing each edge with a pair of doubly directed edges, where each edge has weight $1$.  Equivalent, $\Gamma(G)$ is the digraph of $A(G)$.

Let $G$ be a simple graph on vertex set $[n]$ and $A$ its adjacency matrix.  Then 
\[A\bone = (d_1,\ldots,d_n)\trans,\]
where $\bone\in\mathbb{R}^n$ is the all-ones vector and $d_i$ is the degree of vertex $i$ in $G$.  On the other hand, the $i,j$-entry of $A^r$ is the number of walks from $i$ to $j$ of length $r$.  
  
\section*{Problems}
For the following problems, $G$ is a graph on the vertex set $[n]$ and $A$ is its adjacency matrix.
\begin{enumerate}
\setlength\itemsep{2em}
\item Draw a graph and find its adjacency matrix.
\item If $G$ is $k$-regular, what can you say about $A\bone$?  Is there any obvious eigenvalue and eigenvector?
\item Let $\alpha\subseteq[n]$.  Explain the meaning of $A\bone_\alpha$, where $\bone_\alpha$ is the characteristic vector of $\alpha$ that is $1$ on $\alpha$ and $0$ otherwise.  
\item What is the meaning of the diagonal entries of $A^2$?  Describe $\tr(A^2)$ using the number of edges $m$ of $G$.
\item What is the meaning of the diagonal entries of $A^3$?  Describe $\tr(A^3)$ using the number of triangles $t$ of $G$.
\item Prove by induction that the $i,j$-entry of $A^r$ is the number of walks from $i$ to $j$ of length $r$.
\end{enumerate}

% \newpage
% \section*{Questions to ponder}
% \begin{enumerate}
% \item 
% \end{enumerate}

\end{document}
