\documentclass{article}

%%%PACKAGE
\usepackage{amsmath,amssymb}
\usepackage{amsthm}
%% \usepackage[pdftex,bookmarks=true]{hyperref}
\usepackage{cite}
%% \usepackage{enumerate}
% \usepackage{url}
% \usepackage{hyperref}
% \usepackage{systeme}
%\usepackage{siunitx}
% \usepackage{multicol}
% \usepackage{systeme}

% for drawing graphs
\usepackage{tikz}
% \tikzset{every picture/.style={thick}}
\tikzset{every node/.style={draw, circle, inner sep = 2pt}}
\usetikzlibrary{arrows}

% for margins
\usepackage[margin=1in]{geometry}

% for font
\usepackage{euler}
\usepackage[OT1]{eulervm}
\renewcommand{\rmdefault}{pplx}

% \setlength{\parindent}{0pt}  %no indenting

% MACROS
\newcommand{\trans}{^\top}
\newcommand{\adj}{^{\rm adj}}
\newcommand{\cof}{^{\rm cof}}
\newcommand{\inp}[2]{\left\langle#1,#2\right\rangle}
\newcommand{\dunion}{\mathbin{\dot\cup}}
\newcommand{\bzero}{\mathbf{0}}
\newcommand{\bone}{\mathbf{1}}
\newcommand{\ba}{\mathbf{a}}
\newcommand{\bb}{\mathbf{b}}
\newcommand{\bp}{\mathbf{p}}
\newcommand{\bq}{\mathbf{q}}
\newcommand{\bx}{\mathbf{x}}
\newcommand{\by}{\mathbf{y}}
\newcommand{\bz}{\mathbf{z}}
\newcommand{\bu}{\mathbf{u}}
\newcommand{\bv}{\mathbf{v}}
\newcommand{\bw}{\mathbf{w}}
\newcommand{\tr}{\operatorname{tr}}
\newcommand{\nul}{\operatorname{null}}
\newcommand{\rank}{\operatorname{rank}}
%\newcommand{\ker}{\operatorname{ker}}
\newcommand{\range}{\operatorname{range}}
\newcommand{\Col}{\operatorname{Col}}
\newcommand{\Row}{\operatorname{Row}}
\newcommand{\spec}{\operatorname{spec}}
\newcommand{\vspan}{\operatorname{span}}
% \newenvironment{sol}{\medskip\noindent {\bf Solution.}}{\newpage}
\newcommand{\mystrut}{\rule[-.5\baselineskip]{0pt}{2\baselineskip}}
% \newcommand{\mul}{\operatorname{mul}}
\newcommand{\even}{\operatorname{even}}
\newcommand{\sgn}{\operatorname{sgn}}
\newcommand{\iner}{\operatorname{iner}}

%%%COMMENT
\usepackage{soul}
\usepackage{cancel}
\newcommand{\rbf}[1]{\textbf{\color{red}#1}}

%%%THEOREM
\newtheorem{theorem}{Theorem}[section]
\newtheorem{lemma}[theorem]{Lemma}
\newtheorem{proposition}[theorem]{Proposition}
\newtheorem{corollary}[theorem]{Corollary}

\theoremstyle{definition}
\newtheorem{definition}[theorem]{Definition}
\newtheorem{observation}[theorem]{Observation}
\newtheorem{remark}[theorem]{Remark}
\newtheorem{example}[theorem]{Example}
\newtheorem{notation}[theorem]{Notation}
\newtheorem{question}[theorem]{Question}

% for title
\title{Characteristic polynomial}
\date{\vspace{-1cm}}
\begin{document}
\maketitle
\large

Let $G$ be a graph on $n$ vertices and $A$ its adjacency matrix.  The \emph{adjacency characteristic polynomial} of $G$ is 
\[\det(A - xI) = s_0(-x)^n + s_1(-x)^{n-1} + \cdots + s_n.\]
Each of $s_k$ can be calculated by combinatorial properties of $\Gamma(G)$ as
\[s_k = \sum_{H\in\mathfrak{E}_k(\Gamma(G))} \sgn(H)w(H) = \sum_{H\in\mathfrak{E}_k(\Gamma(G))}(-1)^{k+c(H)}w(H).\]
All we need to do is to translate the digraph setting into the setting of simple graph $G$.  

Let $\alpha\subseteq[n]$ with $|\alpha| = k$.  An elementary subgraph of $\Gamma(G)$ on $\alpha$ corresponds to a subgraph of $G$ on $\alpha$ whose components are either $K_2$ or a cycle.  A such subgraph is called an \emph{elementary subgraph of $G$ on $\alpha$}.  Let $\mathfrak{E}_\alpha(G)$ be the set of all elementary subgraph of $G$ on $\alpha$ and define $\mathfrak{E}_k(G) = \bigcup_{\substack{\alpha\subseteq [n]\\|\alpha| = k}}\mathfrak{E}_\alpha(G)$.    

For any $H\in\mathfrak{G}_\alpha(G)$, let $c(H)$ be the number of components of $H$, $c_2(H)$ be the number of components that is $K_2$, and $c_3(H)$ be the number of components that is a cycle (of length $\geq 3$).  Thus, each $H$ corresponds to $2^{c_3(H)}$ elementary subgraphs of $\Gamma$ on $\alpha$, each has weight $1$ and sign $(-1)^{k + c(H)}$.  Therefore, 
\[s_k = \sum_{H\in\mathfrak{E}_k(\Gamma(G))} \sgn(H)w(H) = \sum_{H\in\mathfrak{E}_k(G)}(-1)^{k+c(H)}2^{c_3(H)}.\]

In particular, 
\[\begin{aligned}
s_0 &= 1, \\
s_1 &= 0, \\
s_2 &= -m, \\
s_3 &= 2t,
\end{aligned}\]
where $m$ is the number of edges on $G$ and $t$ is the number of triangles on $G$.

On the other hand, 
\[\begin{aligned}
\tr(A) &= 0 = s_1, \\
\tr(A^2) &= 2m = -2s_2, \\
\tr(A^3) &= 6t = 3s_3.
\end{aligned}\]

\section*{Problems}
\begin{enumerate}
\setlength\itemsep{2em}
\item Pick a simple graph on $4$ vertices, list all elements in $\mathfrak{E}_k(G)$ for each $k = 0,1,2,3,4$, and calculate the adjacency characteristic polynomial of $G$ using the elementary subgraphs.
\item Use the formula for $s_k$ to derive $s_2 = -m$.  
\item Use the formula for $s_k$ to derive $s_3 = 2t$.  
\item The \emph{girth} of a simple graph is the length of its shortest cycle.  If a graph $G$ has girth $g$, what can you say about its adjacency characteristic polynomial?
\item Let $A$ be a real symmetric matric with eigenvalues $\lambda_1,\ldots, \lambda_n$.  Refresh yourself with the formulas of $s_0,\ldots, s_3$ and $\tr(A),\tr(A^2),\tr(A^3)$ using the eigenvalues.
\end{enumerate}

% \newpage
% \section*{Questions to ponder}
% \begin{enumerate}
% \item 
% \end{enumerate}

\end{document}
