\documentclass{article}

%%%PACKAGE
\usepackage{amsmath,amssymb}
\usepackage{amsthm}
%% \usepackage[pdftex,bookmarks=true]{hyperref}
\usepackage{cite}
%% \usepackage{enumerate}
% \usepackage{url}
% \usepackage{hyperref}
% \usepackage{systeme}
%\usepackage{siunitx}
% \usepackage{multicol}
% \usepackage{systeme}

% for drawing graphs
\usepackage{tikz}
% \tikzset{every picture/.style={thick}}
\tikzset{every node/.style={draw, circle, inner sep = 2pt}}
\usetikzlibrary{arrows}

% for margins
\usepackage[margin=1in]{geometry}

% for font
\usepackage{euler}
\usepackage[OT1]{eulervm}
\renewcommand{\rmdefault}{pplx}

% \setlength{\parindent}{0pt}  %no indenting

% MACROS
\newcommand{\trans}{^\top}
\newcommand{\adj}{^{\rm adj}}
\newcommand{\cof}{^{\rm cof}}
\newcommand{\inp}[2]{\left\langle#1,#2\right\rangle}
\newcommand{\dunion}{\mathbin{\dot\cup}}
\newcommand{\bzero}{\mathbf{0}}
\newcommand{\bone}{\mathbf{1}}
\newcommand{\ba}{\mathbf{a}}
\newcommand{\bb}{\mathbf{b}}
\newcommand{\bp}{\mathbf{p}}
\newcommand{\bq}{\mathbf{q}}
\newcommand{\bx}{\mathbf{x}}
\newcommand{\by}{\mathbf{y}}
\newcommand{\bz}{\mathbf{z}}
\newcommand{\bu}{\mathbf{u}}
\newcommand{\bv}{\mathbf{v}}
\newcommand{\bw}{\mathbf{w}}
\newcommand{\tr}{\operatorname{tr}}
\newcommand{\nul}{\operatorname{null}}
\newcommand{\rank}{\operatorname{rank}}
%\newcommand{\ker}{\operatorname{ker}}
\newcommand{\range}{\operatorname{range}}
\newcommand{\Col}{\operatorname{Col}}
\newcommand{\Row}{\operatorname{Row}}
\newcommand{\spec}{\operatorname{spec}}
\newcommand{\vspan}{\operatorname{span}}
% \newenvironment{sol}{\medskip\noindent {\bf Solution.}}{\newpage}
\newcommand{\mystrut}{\rule[-.5\baselineskip]{0pt}{2\baselineskip}}
% \newcommand{\mul}{\operatorname{mul}}
\newcommand{\even}{\operatorname{even}}
\newcommand{\sgn}{\operatorname{sgn}}
\newcommand{\iner}{\operatorname{iner}}

%%%COMMENT
\usepackage{soul}
\usepackage{cancel}
\newcommand{\rbf}[1]{\textbf{\color{red}#1}}

%%%THEOREM
\newtheorem{theorem}{Theorem}[section]
\newtheorem{lemma}[theorem]{Lemma}
\newtheorem{proposition}[theorem]{Proposition}
\newtheorem{corollary}[theorem]{Corollary}

\theoremstyle{definition}
\newtheorem{definition}[theorem]{Definition}
\newtheorem{observation}[theorem]{Observation}
\newtheorem{remark}[theorem]{Remark}
\newtheorem{example}[theorem]{Example}
\newtheorem{notation}[theorem]{Notation}
\newtheorem{question}[theorem]{Question}

% for title
\title{Perron--Frobenius theorem}
\date{\vspace{-1cm}}
\begin{document}
\maketitle
\large

Let $A$ be an $n\times n$ real (not necessarily symmetric) matrix.  We say $A > 0$ if every entry of $A$ is positive, while $A\geq 0$ is similarly defined.  

If there is a permutation matrix $P$ such that 
\[PAP\trans = \begin{bmatrix}
 B & C \\
 O & D
\end{bmatrix}\]
for some nontrivial square matrices $B$ and $D$, then $A$ is said to be \emph{reducible}.  Otherwise, $A$ is called \emph{irreducible}.  Let $\Gamma$ be the digraph of $A$.  Then $A$ is irreducible if and only if $\Gamma$ is strongly connected.  On the other hand, the \emph{period} of $A$ is the greatest common divisor of the length of all closed walk on $\Gamma$.  

Let $A$ be an $n\times n$ irreducible matrix such that $A\geq 0$.  
The \textbf{Perron--Frobenius theorem} states (at least) the following:
\begin{enumerate}
\item There is an eigenvalue $\rho$ of $A$ that is real and $|\lambda|\leq\rho$ for any eigenvalue of $A$.
\item The algebraic and geometric multiplicity of $\rho$ is $1$.  Moreover, there is an eigenvector of $A$ with respect to $\rho$ that is entrywisely positive.
\item If $\bu$ is an eigenvector of $A$ that is 
entrywisely positive, then $\bu$ is an eigenvector with respect to $\rho$.
\item If $A$ has period $k$, then there are exactly $k$ eigenvalues $\lambda$ of $A$ with $|\lambda| = \rho$.  
\end{enumerate}

Let $G$ be a graph and $A$ its adjacency matrix.  Then the \emph{spectral radius} of $G$ is the largest eigenvalue of $A$, denoted by $\rho(G)$.

\section*{Problems}
\begin{enumerate}
\setlength\itemsep{2em}
\item Draw a digraph that is strongly connected.  Draw a digraph that is weakly connected but not strongly connected.
\item Find a matrix $A \geq 0$ that is reducible.  Find a matrix $A \geq 0$ that is irreducible.
\item If $A$ is the adjacency matrix of $G$, how can we determine if $A$ is irreducible or not by $G$?
\item If $A$ is the adjacency matrix of $G$, what are the possible periods of $A$? 
\item For each of the following condition, pick a graph $G$ and calculate its spectral radius and the corresponding eigenvector.  Here $A$ is the adjacency matrix of $G$.
\begin{enumerate}
\item The matrix $A$ is irreducible and has period $1$.
\item The matrix $A$ is irreducible and has period $2$.
\item The matrix $A$ is not irreducible.
\end{enumerate}
\end{enumerate}

% \newpage
% \section*{Questions to ponder}
% \begin{enumerate}
% \item 
% \end{enumerate}

\end{document}
