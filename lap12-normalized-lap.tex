\documentclass{article}

%%%PACKAGE
\usepackage{amsmath,amssymb}
\usepackage{amsthm}
%% \usepackage[pdftex,bookmarks=true]{hyperref}
\usepackage{cite}
%% \usepackage{enumerate}
% \usepackage{url}
% \usepackage{hyperref}
% \usepackage{systeme}
%\usepackage{siunitx}
% \usepackage{multicol}
% \usepackage{systeme}

% for drawing graphs
\usepackage{tikz}
% \tikzset{every picture/.style={thick}}
\tikzset{every node/.style={draw, circle, inner sep = 2pt}}
\usetikzlibrary{arrows}

% for margins
\usepackage[margin=1in]{geometry}

% for font
\usepackage{euler}
\usepackage[OT1]{eulervm}
\renewcommand{\rmdefault}{pplx}

% \setlength{\parindent}{0pt}  %no indenting

% MACROS
\newcommand{\trans}{^\top}
\newcommand{\adj}{^{\rm adj}}
\newcommand{\cof}{^{\rm cof}}
\newcommand{\inp}[2]{\left\langle#1,#2\right\rangle}
\newcommand{\dunion}{\mathbin{\dot\cup}}
\newcommand{\bzero}{\mathbf{0}}
\newcommand{\bone}{\mathbf{1}}
\newcommand{\ba}{\mathbf{a}}
\newcommand{\bb}{\mathbf{b}}
\newcommand{\bd}{\mathbf{d}}
\newcommand{\be}{\mathbf{e}}
\newcommand{\bp}{\mathbf{p}}
\newcommand{\bq}{\mathbf{q}}
\newcommand{\bx}{\mathbf{x}}
\newcommand{\by}{\mathbf{y}}
\newcommand{\bz}{\mathbf{z}}
\newcommand{\bu}{\mathbf{u}}
\newcommand{\bv}{\mathbf{v}}
\newcommand{\bw}{\mathbf{w}}
\newcommand{\tr}{\operatorname{tr}}
\newcommand{\nul}{\operatorname{null}}
\newcommand{\rank}{\operatorname{rank}}
%\newcommand{\ker}{\operatorname{ker}}
\newcommand{\range}{\operatorname{range}}
\newcommand{\Col}{\operatorname{Col}}
\newcommand{\Row}{\operatorname{Row}}
\newcommand{\spec}{\operatorname{spec}}
\newcommand{\vspan}{\operatorname{span}}
% \newenvironment{sol}{\medskip\noindent {\bf Solution.}}{\newpage}
\newcommand{\mystrut}{\rule[-.5\baselineskip]{0pt}{2\baselineskip}}
% \newcommand{\mul}{\operatorname{mul}}
\newcommand{\even}{\operatorname{even}}
\newcommand{\sgn}{\operatorname{sgn}}
\newcommand{\iner}{\operatorname{iner}}
\newcommand{\rL}{\mathring{L}}
\newcommand{\diag}{\operatorname{diag}}

%%%COMMENT
\usepackage{soul}
\usepackage{cancel}
\newcommand{\rbf}[1]{\textbf{\color{red}#1}}

%%%THEOREM
\newtheorem{theorem}{Theorem}[section]
\newtheorem{lemma}[theorem]{Lemma}
\newtheorem{proposition}[theorem]{Proposition}
\newtheorem{corollary}[theorem]{Corollary}

\theoremstyle{definition}
\newtheorem{definition}[theorem]{Definition}
\newtheorem{observation}[theorem]{Observation}
\newtheorem{remark}[theorem]{Remark}
\newtheorem{example}[theorem]{Example}
\newtheorem{notation}[theorem]{Notation}
\newtheorem{question}[theorem]{Question}

% for title
\title{Normalized Laplacian matrix}
\date{\vspace{-1cm}}
\begin{document}
\maketitle
\large

Let $G$ be a simple graph, $L$ its Laplacian matrix, and $D = \diag(d_i, \ldots, d_n)$ the degree matrix.  We define 
\[D^{-1/2} = \diag(\frac{1}{\sqrt{d_1}}, \cdots, \frac{1}{\sqrt{d_n}}),\]
where $\frac{1}{\sqrt{d_i}}$ is replaced with $0$ if $d_i = 0$.  Thus, the \emph{normalized Laplacian matrix} of $G$ is defined as $\mathcal{L}(G) = D^{-1/2}LD^{-1/2}$.  Note that the diagonal entry of a Laplacian matrix is always $1$ unless the corresponding vertex is isolated.

% Let $G$ be a graph on $n$ vertices and $\mathcal{L}$ its normalized Laplacian matrix.  Let $\bx = (x_1,\ldots,x_n)\trans \in \mathbb{R}^n$.  Note that $\mathcal{L}$ can be viewed as the weighted Laplacian matrix of $G$ with weight 
% \[w(i,j) = \frac{1}{\sqrt{d_idj}}.\]  
% Thus, the quadratic form is 
The $i,j$-entry of $\mathcal{L}$ is 
\[
\begin{cases}
1 & \text{if }i = j\text{ is not an isolated vertx},\\
-\frac{1}{\sqrt{d_idj}} & \text{if }\{i,j\}\in E(G),\\
0 & \text{otherwise}.
\end{cases}
\]
The quadratic form is.
\[\bx\trans\mathcal{L}\bx = \sum_{\{i,j\}\in E(G)} \left(\frac{x_i}{\sqrt{d_i}} - \frac{x_j}{\sqrt{d_j}}\right)^2.\]
A normalized Laplacian matrix has the following properties:  
\begin{itemize}
\item $\mathcal{L}\bd^{1/2} = \bzero$, where $\bd^{1/2} = (\sqrt{d_1}, \ldots, \sqrt{d_n})$. 
\item $\mathcal{L}$ is positive semidefinite, and $\nul(\mathcal{L})$ is the number of components of $G$.
\item the eigenvalues of $\mathcal{L}$ are in $[0,2]$, and the multiplicity of $2$ is the number of components of $G$ that is a bipartite graph.
\item The random walk matrix $D^{-1}A$ is similar to $I - \mathcal{L}$, so $D^{-1}A$ and $\mathcal{L}$ share the same eigenvectors, and $\lambda$ is an eigenvalue of $D^{-1}A$ if and only if $1-\lambda$ is an eigenvalue of $\mathcal{L}$.  
\end{itemize}

More details on this topic can be found in \cite{FRKChung96}.

\begin{thebibliography}{9}
\bibitem{FRKChung96}
Fan R.~K. Chung.
\newblock {\em Spectral {G}raph {T}heory}.
\newblock American Mathematical Society, 1996.
\end{thebibliography}

\section*{Problems}
\begin{enumerate}
\setlength\itemsep{2em}
\item Find the eigenvalues and eigenvectors of $\mathcal{L}(K_4)$.
\item Find the eigenvalues and eigenvectors of $\mathcal{L}(K_{2,3})$.
\item Prove the formula of the quadratic form.  
\item If $G$ is $k$-regular, how can you obtain the eigenvalues of $\mathcal{L}(G)$ from the eigenvalues of $L(G)$.  
\item Prove that every eigenvalue of $\mathcal{L}(G)$ is at most $2$ and the multiplicity of $2$ is the number of components of $G$ that is a bipartite graph.
\item Explain why $D^{-1}A$ and $\mathcal{L}$ are similar.
\end{enumerate}

% \newpage
% \section*{Questions to ponder}
% \begin{enumerate}
% \item 
% \end{enumerate}

\end{document}
