\documentclass{article}

%%%PACKAGE
\usepackage{amsmath,amssymb}
\usepackage{amsthm}
%% \usepackage[pdftex,bookmarks=true]{hyperref}
\usepackage{cite}
%% \usepackage{enumerate}
% \usepackage{url}
% \usepackage{hyperref}
% \usepackage{systeme}
%\usepackage{siunitx}
% \usepackage{multicol}
% \usepackage{systeme}

% for drawing graphs
\usepackage{tikz}
% \tikzset{every picture/.style={thick}}
\tikzset{every node/.style={draw, circle, inner sep = 2pt}}
\usetikzlibrary{arrows}

% for margins
\usepackage[margin=1in]{geometry}

% for font
\usepackage{euler}
\usepackage[OT1]{eulervm}
\renewcommand{\rmdefault}{pplx}

% \setlength{\parindent}{0pt}  %no indenting

% MACROS
\newcommand{\trans}{^\top}
\newcommand{\adj}{^{\rm adj}}
\newcommand{\cof}{^{\rm cof}}
\newcommand{\inp}[2]{\left\langle#1,#2\right\rangle}
\newcommand{\dunion}{\mathbin{\dot\cup}}
\newcommand{\bzero}{\mathbf{0}}
\newcommand{\bone}{\mathbf{1}}
\newcommand{\ba}{\mathbf{a}}
\newcommand{\bb}{\mathbf{b}}
\newcommand{\bp}{\mathbf{p}}
\newcommand{\bq}{\mathbf{q}}
\newcommand{\bx}{\mathbf{x}}
\newcommand{\by}{\mathbf{y}}
\newcommand{\bz}{\mathbf{z}}
\newcommand{\bu}{\mathbf{u}}
\newcommand{\bv}{\mathbf{v}}
\newcommand{\bw}{\mathbf{w}}
\newcommand{\tr}{\operatorname{tr}}
\newcommand{\nul}{\operatorname{null}}
\newcommand{\rank}{\operatorname{rank}}
%\newcommand{\ker}{\operatorname{ker}}
\newcommand{\range}{\operatorname{range}}
\newcommand{\Col}{\operatorname{Col}}
\newcommand{\Row}{\operatorname{Row}}
\newcommand{\spec}{\operatorname{spec}}
\newcommand{\vspan}{\operatorname{span}}
% \newenvironment{sol}{\medskip\noindent {\bf Solution.}}{\newpage}
\newcommand{\mystrut}{\rule[-.5\baselineskip]{0pt}{2\baselineskip}}
% \newcommand{\mul}{\operatorname{mul}}

%%%COMMENT
\usepackage{soul}
\usepackage{cancel}
\newcommand{\rbf}[1]{\textbf{\color{red}#1}}

%%%THEOREM
\newtheorem{theorem}{Theorem}[section]
\newtheorem{lemma}[theorem]{Lemma}
\newtheorem{proposition}[theorem]{Proposition}
\newtheorem{corollary}[theorem]{Corollary}

\theoremstyle{definition}
\newtheorem{definition}[theorem]{Definition}
\newtheorem{observation}[theorem]{Observation}
\newtheorem{remark}[theorem]{Remark}
\newtheorem{example}[theorem]{Example}
\newtheorem{notation}[theorem]{Notation}
\newtheorem{question}[theorem]{Question}

% for title
\title{Matrix-vector multiplication}
\date{\vspace{-1cm}}
\begin{document}
\maketitle
\large

Let $A = \begin{bmatrix} a_{ij}\end{bmatrix}$ be an $n\times n$ matrix and $\bx = (x_1,\ldots, x_n)\trans$ a column vector in $\mathbb{R}^n$.  Then $A\bx$ is another vector in $\mathbb{R}^n$ whose $i$-th entry is 
\[\sum_{k\in[n]}a_{ik}x_k.\]
Here $[n] = \{1,\ldots, n\}$.  

Based on an $n\times n$ square matrix $A$, we may define a digraph $\Gamma$ whose vertex set is $V(\Gamma) = [n]$ and whose edge set is 
\[E(\Gamma) = \{(i,j) : a_{i,j}\neq 0\}.\]
The digraph $\Gamma$ is called the \emph{digraph of $A$}.  (Note that digraphs allow loops.)  The \emph{out-neighborhood} of a vertex $i$ of $\Gamma$ is 
\[N^+_\Gamma(i) = \{k : (i,k) \in E(\Gamma)\}.\]
We often write $N^+(i)$ when the digraph is clear in the context.
Thus, the matrix-vector multiplication can be simplified such that the $i$-th entry of $A\bx$ is 
\[\sum_{k\in N^+_\Gamma(i)}a_{ik}x_k.\]
This is particularly useful when the matrix is sparse.

\section*{Problems}
\begin{enumerate}
\setlength\itemsep{2em}
\item Write a $4\times 4$ matrix $A$ and a vector $\bx$ in $\mathbb{R}^4$.  Draw the digraph of $A$.  Calculate $A\bx$ by the digraph.
\item When the matrix $A$ is symmetric, how would you define the ``simple'' graph of $A$?  How does the matrix-vector multiplication work?  What if $A$ is a symmetric matrix with zero diagonal?  (You might need to define the closed and open neighborhood on a simple graph.)
\item A \emph{permutation matrix} is a matrix where there is exactly a $1$ on each row and on each column.  Pick a permutation matrix and draw its digraph.  Describe some features of the digraph of a permutation matrix.
\end{enumerate}

% \newpage
% \section*{Questions to ponder}
% \begin{enumerate}
% \item 
% \end{enumerate}

\end{document}
