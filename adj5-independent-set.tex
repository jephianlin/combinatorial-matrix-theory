\documentclass{article}

%%%PACKAGE
\usepackage{amsmath,amssymb}
\usepackage{amsthm}
%% \usepackage[pdftex,bookmarks=true]{hyperref}
\usepackage{cite}
%% \usepackage{enumerate}
% \usepackage{url}
% \usepackage{hyperref}
% \usepackage{systeme}
%\usepackage{siunitx}
% \usepackage{multicol}
% \usepackage{systeme}

% for drawing graphs
\usepackage{tikz}
% \tikzset{every picture/.style={thick}}
\tikzset{every node/.style={draw, circle, inner sep = 2pt}}
\usetikzlibrary{arrows}

% for margins
\usepackage[margin=1in]{geometry}

% for font
\usepackage{euler}
\usepackage[OT1]{eulervm}
\renewcommand{\rmdefault}{pplx}

% \setlength{\parindent}{0pt}  %no indenting

% MACROS
\newcommand{\trans}{^\top}
\newcommand{\adj}{^{\rm adj}}
\newcommand{\cof}{^{\rm cof}}
\newcommand{\inp}[2]{\left\langle#1,#2\right\rangle}
\newcommand{\dunion}{\mathbin{\dot\cup}}
\newcommand{\bzero}{\mathbf{0}}
\newcommand{\bone}{\mathbf{1}}
\newcommand{\ba}{\mathbf{a}}
\newcommand{\bb}{\mathbf{b}}
\newcommand{\bp}{\mathbf{p}}
\newcommand{\bq}{\mathbf{q}}
\newcommand{\bx}{\mathbf{x}}
\newcommand{\by}{\mathbf{y}}
\newcommand{\bz}{\mathbf{z}}
\newcommand{\bu}{\mathbf{u}}
\newcommand{\bv}{\mathbf{v}}
\newcommand{\bw}{\mathbf{w}}
\newcommand{\tr}{\operatorname{tr}}
\newcommand{\nul}{\operatorname{null}}
\newcommand{\rank}{\operatorname{rank}}
%\newcommand{\ker}{\operatorname{ker}}
\newcommand{\range}{\operatorname{range}}
\newcommand{\Col}{\operatorname{Col}}
\newcommand{\Row}{\operatorname{Row}}
\newcommand{\spec}{\operatorname{spec}}
\newcommand{\vspan}{\operatorname{span}}
% \newenvironment{sol}{\medskip\noindent {\bf Solution.}}{\newpage}
\newcommand{\mystrut}{\rule[-.5\baselineskip]{0pt}{2\baselineskip}}
% \newcommand{\mul}{\operatorname{mul}}
\newcommand{\even}{\operatorname{even}}
\newcommand{\sgn}{\operatorname{sgn}}
\newcommand{\iner}{\operatorname{iner}}

%%%COMMENT
\usepackage{soul}
\usepackage{cancel}
\newcommand{\rbf}[1]{\textbf{\color{red}#1}}

%%%THEOREM
\newtheorem{theorem}{Theorem}[section]
\newtheorem{lemma}[theorem]{Lemma}
\newtheorem{proposition}[theorem]{Proposition}
\newtheorem{corollary}[theorem]{Corollary}

\theoremstyle{definition}
\newtheorem{definition}[theorem]{Definition}
\newtheorem{observation}[theorem]{Observation}
\newtheorem{remark}[theorem]{Remark}
\newtheorem{example}[theorem]{Example}
\newtheorem{notation}[theorem]{Notation}
\newtheorem{question}[theorem]{Question}

% for title
\title{Independent set}
\date{\vspace{-1cm}}
\begin{document}
\maketitle
\large

Let $G$ be a graph on $n$ vertices and $A$ its adjacency matrix.  An \emph{independent set} of $G$ is a set $S$ of vertices such that the induced subgraph $G[S]$ contains no edge.  Equivalently, $S\subseteq[n]$ is an independent set of $G$ if and only if $A[S] = O$.  The \emph{independence number} of $G$ is the cardinality of the largest independent set of $G$, denoted by $\alpha(G)$.  

Let $G$ be a graph on $n$ vertices, $A$ its adjacency matrix, and $\lambda_1\leq \cdots \leq \lambda_n$ the eigenvalues of $A$.  Let $k = \alpha(G)$.  Then $A$ contains a $k\times k$ principal submatrix that is a zero matrix, so its eigenvalues are 
\[\mu_1 = \cdots = \mu_k = 0.\]
By the interlacing property, we know 
\[\lambda_k \leq 0 \text{ and } \lambda_{n - (k-1)} \geq 0.\]
Equivalently, $n_-(A) + n_0(A) \geq k$ and $n_+(A) + n_0(A) \geq k$.  Therefore, 
\[\alpha(G) \leq \min\{n_-(A) + n_0(A), n_+(A) + n_0(A)\}.\]
This bound is called \textbf{Cvetkovi\'c's inertia bound}.

\section*{Problems}
\begin{enumerate}
\setlength\itemsep{2em}
\item Pick a graph, calculate its independence number and its inertia, and then check if Cvetkovi\'c's inertia bound is tight.   
\item Find two graphs such that the inertia bound is tight for one graph but is not tight for the other.  
\item Suppose $G$ has an induced subgraph that is isomorphic to $K_{m,n}$ with $k = m + n$.  Prove that $k - 1 \leq n_-(A) + n_0(A)$, where $A$ is the adjacency matrix of $G$.
\end{enumerate}

% \newpage
% \section*{Questions to ponder}
% \begin{enumerate}
% \item 
% \end{enumerate}

\end{document}
