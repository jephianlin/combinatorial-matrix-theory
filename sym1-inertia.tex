\documentclass{article}

%%%PACKAGE
\usepackage{amsmath,amssymb}
\usepackage{amsthm}
%% \usepackage[pdftex,bookmarks=true]{hyperref}
\usepackage{cite}
%% \usepackage{enumerate}
% \usepackage{url}
% \usepackage{hyperref}
% \usepackage{systeme}
%\usepackage{siunitx}
% \usepackage{multicol}
% \usepackage{systeme}

% for drawing graphs
\usepackage{tikz}
% \tikzset{every picture/.style={thick}}
\tikzset{every node/.style={draw, circle, inner sep = 2pt}}
\usetikzlibrary{arrows}

% for margins
\usepackage[margin=1in]{geometry}

% for font
\usepackage{euler}
\usepackage[OT1]{eulervm}
\renewcommand{\rmdefault}{pplx}

% \setlength{\parindent}{0pt}  %no indenting

% MACROS
\newcommand{\trans}{^\top}
\newcommand{\adj}{^{\rm adj}}
\newcommand{\cof}{^{\rm cof}}
\newcommand{\inp}[2]{\left\langle#1,#2\right\rangle}
\newcommand{\dunion}{\mathbin{\dot\cup}}
\newcommand{\bzero}{\mathbf{0}}
\newcommand{\bone}{\mathbf{1}}
\newcommand{\ba}{\mathbf{a}}
\newcommand{\bb}{\mathbf{b}}
\newcommand{\bp}{\mathbf{p}}
\newcommand{\bq}{\mathbf{q}}
\newcommand{\bx}{\mathbf{x}}
\newcommand{\by}{\mathbf{y}}
\newcommand{\bz}{\mathbf{z}}
\newcommand{\bu}{\mathbf{u}}
\newcommand{\bv}{\mathbf{v}}
\newcommand{\bw}{\mathbf{w}}
\newcommand{\tr}{\operatorname{tr}}
\newcommand{\nul}{\operatorname{null}}
\newcommand{\rank}{\operatorname{rank}}
%\newcommand{\ker}{\operatorname{ker}}
\newcommand{\range}{\operatorname{range}}
\newcommand{\Col}{\operatorname{Col}}
\newcommand{\Row}{\operatorname{Row}}
\newcommand{\spec}{\operatorname{spec}}
\newcommand{\vspan}{\operatorname{span}}
% \newenvironment{sol}{\medskip\noindent {\bf Solution.}}{\newpage}
\newcommand{\mystrut}{\rule[-.5\baselineskip]{0pt}{2\baselineskip}}
% \newcommand{\mul}{\operatorname{mul}}
\newcommand{\even}{\operatorname{even}}
\newcommand{\sgn}{\operatorname{sgn}}
\newcommand{\iner}{\operatorname{iner}}

%%%COMMENT
\usepackage{soul}
\usepackage{cancel}
\newcommand{\rbf}[1]{\textbf{\color{red}#1}}

%%%THEOREM
\newtheorem{theorem}{Theorem}[section]
\newtheorem{lemma}[theorem]{Lemma}
\newtheorem{proposition}[theorem]{Proposition}
\newtheorem{corollary}[theorem]{Corollary}

\theoremstyle{definition}
\newtheorem{definition}[theorem]{Definition}
\newtheorem{observation}[theorem]{Observation}
\newtheorem{remark}[theorem]{Remark}
\newtheorem{example}[theorem]{Example}
\newtheorem{notation}[theorem]{Notation}
\newtheorem{question}[theorem]{Question}

% for title
\title{Inertia and congruence}
\date{\vspace{-1cm}}
\begin{document}
\maketitle
\large

From now on, we will focus on real symmetric matrices.  (Most of the theorems also apply to complex Hermitian matrices.)

Let $A = \begin{bmatrix} a_{ij} \end{bmatrix}$ be an $n\times n$ real symmetric matrix.  By the \textbf{spectral theorem}, $A$ is diagonalizable through an \textbf{orthonormal} eigenbasis $\{\bu_1,\ldots,\bu_n\}$ and the corresponding eigenvalues $\{\lambda_1,\ldots,\lambda_n\}$ are \textbf{real} numbers.  Since the eigenvalues are real, we may assume $\lambda_1\leq \cdots \leq \lambda_n$.  

The \emph{inertia} of $A$ is a tuple 
\[\iner(A) = (n_+(A), n_-(A), n_0(A)),\]
where $n_+(A)$, $n_-(A)$, and $n_0(A)$ are the number of positive, negative, and zero eigenvalues of $A$.  

Two real symmetric matrices $A$ and $B$ are \emph{congruent} if there is an \textbf{invertible} matrix $Q$ such that $B = Q\trans AQ$.  According to \textbf{Sylvester's law of inertia}, two congruent matrices have the same inertia.

Recall that every invertible matrix can be decomposed into the product of a sequence of elementary matrices, which correspond to a sequence of row operations.  Therefore, $A$ and $B$ are congruent if and only if $B$ can be obtained from $A$ by a sequence of symmetric row/column operations.  Indeed, every real symmetric matrix is congruent to the following canonical form 
\[\begin{bmatrix}
 I_p & & \\
  & -I_q & \\
  & & O_r 
\end{bmatrix},\]
where $p = n_+(A)$, $q = n_-(A)$, and $r = n_0(A)$.  

\section*{Problems}
\begin{enumerate}
\setlength\itemsep{2em}
\item Write a $3\times 3$ real symmetric matrix.  Diagonalize it, and find its inertia.
\item Use the same matrix you had, and apply symmetric row/column operations to find its inertia.
\item Let $J_n$ be the $n\times n$ all-ones matrix.  Find its inertia.  Find an invertible matrix through which $J_n$ is congruent to its canonical form.
\end{enumerate}

% \newpage
% \section*{Questions to ponder}
% \begin{enumerate}
% \item 
% \end{enumerate}

\end{document}
