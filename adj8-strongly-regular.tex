\documentclass{article}

%%%PACKAGE
\usepackage{amsmath,amssymb}
\usepackage{amsthm}
%% \usepackage[pdftex,bookmarks=true]{hyperref}
\usepackage{cite}
%% \usepackage{enumerate}
% \usepackage{url}
% \usepackage{hyperref}
% \usepackage{systeme}
%\usepackage{siunitx}
% \usepackage{multicol}
% \usepackage{systeme}

% for drawing graphs
\usepackage{tikz}
% \tikzset{every picture/.style={thick}}
\tikzset{every node/.style={draw, circle, inner sep = 2pt}}
\usetikzlibrary{arrows}

% for margins
\usepackage[margin=1in]{geometry}

% for font
\usepackage{euler}
\usepackage[OT1]{eulervm}
\renewcommand{\rmdefault}{pplx}

% \setlength{\parindent}{0pt}  %no indenting

% MACROS
\newcommand{\trans}{^\top}
\newcommand{\adj}{^{\rm adj}}
\newcommand{\cof}{^{\rm cof}}
\newcommand{\inp}[2]{\left\langle#1,#2\right\rangle}
\newcommand{\dunion}{\mathbin{\dot\cup}}
\newcommand{\bzero}{\mathbf{0}}
\newcommand{\bone}{\mathbf{1}}
\newcommand{\ba}{\mathbf{a}}
\newcommand{\bb}{\mathbf{b}}
\newcommand{\bp}{\mathbf{p}}
\newcommand{\bq}{\mathbf{q}}
\newcommand{\bx}{\mathbf{x}}
\newcommand{\by}{\mathbf{y}}
\newcommand{\bz}{\mathbf{z}}
\newcommand{\bu}{\mathbf{u}}
\newcommand{\bv}{\mathbf{v}}
\newcommand{\bw}{\mathbf{w}}
\newcommand{\tr}{\operatorname{tr}}
\newcommand{\nul}{\operatorname{null}}
\newcommand{\rank}{\operatorname{rank}}
%\newcommand{\ker}{\operatorname{ker}}
\newcommand{\range}{\operatorname{range}}
\newcommand{\Col}{\operatorname{Col}}
\newcommand{\Row}{\operatorname{Row}}
\newcommand{\spec}{\operatorname{spec}}
\newcommand{\vspan}{\operatorname{span}}
% \newenvironment{sol}{\medskip\noindent {\bf Solution.}}{\newpage}
\newcommand{\mystrut}{\rule[-.5\baselineskip]{0pt}{2\baselineskip}}
% \newcommand{\mul}{\operatorname{mul}}
\newcommand{\even}{\operatorname{even}}
\newcommand{\sgn}{\operatorname{sgn}}
\newcommand{\iner}{\operatorname{iner}}

%%%COMMENT
\usepackage{soul}
\usepackage{cancel}
\newcommand{\rbf}[1]{\textbf{\color{red}#1}}

%%%THEOREM
\newtheorem{theorem}{Theorem}[section]
\newtheorem{lemma}[theorem]{Lemma}
\newtheorem{proposition}[theorem]{Proposition}
\newtheorem{corollary}[theorem]{Corollary}

\theoremstyle{definition}
\newtheorem{definition}[theorem]{Definition}
\newtheorem{observation}[theorem]{Observation}
\newtheorem{remark}[theorem]{Remark}
\newtheorem{example}[theorem]{Example}
\newtheorem{notation}[theorem]{Notation}
\newtheorem{question}[theorem]{Question}

% for title
\title{Strongly regular graph}
\date{\vspace{-1cm}}
\begin{document}
\maketitle
\large

A simple graph $G$ is \emph{strongly regular} if there are parameters $v,k,\lambda,\mu$ such that \begin{itemize}
\item $G$ has $v$ vertices,
\item $G$ is $k$-regular, 
\item for each pair of adjacent vertices, they have exactly $\lambda$ common neighbors, and 
\item for each pair of non-adjacent vertices, they have exactly $\mu$ common neighbors.
\end{itemize}

Let $G$ be a strongly regular graph with parameters $v,k,\lambda,\mu$ and $A$ its adjacency matrix.  Then 
\[A^2 = kI + \lambda A + \mu(J - I - A),\]
where $J$ is the $v\times v$ all-ones matrix.
Since $G$ is $k$-regular, $A\bone = k\bone$ and 
\[k^2 = k + \lambda k + \mu(v - 1 - k).\]
We may assume other eigenvectors are orthogonal to $\bone$.  Let $\bu$ be an eigenvector with respect to eigenvalue $\theta$ such that $\bu\perp\bone$.  Then 
\[\theta^2 = k + \lambda\theta + \mu(-1 - \theta),\] 
so 
\[\theta = \frac{\lambda - \mu \pm \sqrt{(\mu - \lambda)^2 - 4(\mu - k)}}{2}.\]


\section*{Problems}
\begin{enumerate}
\setlength\itemsep{2em}
% \item Is $K_n$ a strongly regular graph?  If so, what are $v,k,\lambda,\mu$?
\item Is $C_n$ a strongly regular graph?  If so, what are $v,k,\lambda,\mu$?
\item Is the Petersen graph a strongly regular graph?  If so, what are $v,k,\lambda,\mu$?
\item Find the adjacency eigenvalues of the Petersen graph without using the formula.
\item Let $G$ be a strongly regular graph and $i$ a vertex of $G$.  Let $X$ and $Y$ be the vertices that are adjacent and not adjacent to $i$.  Count the number of edges between $X$ and $Y$ and show that
\[k(k - 1 - \lambda) = \mu(v - k - 1).\]
\item Use any online resources if necessary, find the multiplicities of the adjacency eigenvalues of a strongly regular graph.
\end{enumerate}

% \newpage
% \section*{Questions to ponder}
% \begin{enumerate}
% \item 
% \end{enumerate}

\end{document}
