\documentclass{article}

%%%PACKAGE
\usepackage{amsmath,amssymb}
\usepackage{amsthm}
%% \usepackage[pdftex,bookmarks=true]{hyperref}
\usepackage{cite}
%% \usepackage{enumerate}
% \usepackage{url}
% \usepackage{hyperref}
% \usepackage{systeme}
%\usepackage{siunitx}
% \usepackage{multicol}
% \usepackage{systeme}

% for drawing graphs
\usepackage{tikz}
% \tikzset{every picture/.style={thick}}
\tikzset{every node/.style={draw, circle, inner sep = 2pt}}
\usetikzlibrary{arrows}

% for margins
\usepackage[margin=1in]{geometry}

% for font
\usepackage{euler}
\usepackage[OT1]{eulervm}
\renewcommand{\rmdefault}{pplx}

% \setlength{\parindent}{0pt}  %no indenting

% MACROS
\newcommand{\trans}{^\top}
\newcommand{\adj}{^{\rm adj}}
\newcommand{\cof}{^{\rm cof}}
\newcommand{\inp}[2]{\left\langle#1,#2\right\rangle}
\newcommand{\dunion}{\mathbin{\dot\cup}}
\newcommand{\bzero}{\mathbf{0}}
\newcommand{\bone}{\mathbf{1}}
\newcommand{\ba}{\mathbf{a}}
\newcommand{\bb}{\mathbf{b}}
\newcommand{\bp}{\mathbf{p}}
\newcommand{\bq}{\mathbf{q}}
\newcommand{\bx}{\mathbf{x}}
\newcommand{\by}{\mathbf{y}}
\newcommand{\bz}{\mathbf{z}}
\newcommand{\bu}{\mathbf{u}}
\newcommand{\bv}{\mathbf{v}}
\newcommand{\bw}{\mathbf{w}}
\newcommand{\tr}{\operatorname{tr}}
\newcommand{\nul}{\operatorname{null}}
\newcommand{\rank}{\operatorname{rank}}
%\newcommand{\ker}{\operatorname{ker}}
\newcommand{\range}{\operatorname{range}}
\newcommand{\Col}{\operatorname{Col}}
\newcommand{\Row}{\operatorname{Row}}
\newcommand{\spec}{\operatorname{spec}}
\newcommand{\vspan}{\operatorname{span}}
% \newenvironment{sol}{\medskip\noindent {\bf Solution.}}{\newpage}
\newcommand{\mystrut}{\rule[-.5\baselineskip]{0pt}{2\baselineskip}}
% \newcommand{\mul}{\operatorname{mul}}
\newcommand{\even}{\operatorname{even}}
\newcommand{\sgn}{\operatorname{sgn}}
\newcommand{\iner}{\operatorname{iner}}

%%%COMMENT
\usepackage{soul}
\usepackage{cancel}
\newcommand{\rbf}[1]{\textbf{\color{red}#1}}

%%%THEOREM
\newtheorem{theorem}{Theorem}[section]
\newtheorem{lemma}[theorem]{Lemma}
\newtheorem{proposition}[theorem]{Proposition}
\newtheorem{corollary}[theorem]{Corollary}

\theoremstyle{definition}
\newtheorem{definition}[theorem]{Definition}
\newtheorem{observation}[theorem]{Observation}
\newtheorem{remark}[theorem]{Remark}
\newtheorem{example}[theorem]{Example}
\newtheorem{notation}[theorem]{Notation}
\newtheorem{question}[theorem]{Question}

% for title
\title{Quadratic form}
\date{\vspace{-1cm}}
\begin{document}
\maketitle
\large

Let $G$ be a simple graph on $n$ vertices.  The \emph{Laplacian matrix} of $G$ is a matrix $L(G) = \begin{bmatrix} \ell_{ij} \end{bmatrix}$ such that 
\[\ell_{ij} = \begin{cases}
\deg_G(i) & \text{if }i = j, \\
-1 & \text{if }\{i,j\}\in E(G), \\
0 & otherwise.
\end{cases}\]
Since the row sums of $L(G)$ are zero, $L(G)\bone = \bzero$ for any $G$.  

Observe that the Laplacian matrix of $K_2$ is 
\[\begin{bmatrix} 
 1 & -1 \\
 -1 & 1 
\end{bmatrix},\]
and its quadratic form is 
\[\begin{bmatrix} x_1 & x_2 \end{bmatrix} 
\begin{bmatrix} 
 1 & -1 \\
 -1 & 1 
\end{bmatrix}
\begin{bmatrix} x_1 \\ x_2 \end{bmatrix} = x_1^2 - 2x_1x_2 + x_2^2 = (x_1 - x_2)^2.\]
In fact, if $\bx = (x_1,\ldots,x_n)\trans$ and $L = L(G)$, then 
\[\bx\trans L\bx = \sum_{\{i,j\}\in E(G)}(x_i - x_j)^2.\]
Therefore, $L(G)$ is a positive semidefinite matrix for any $G$ and $L\bx = \bzero$ if and only if $\bx\trans L\bx = 0$.

Another way to see this is through the incidence matrix.  Let $G$ be a graph on vertex set $\{v_1,\ldots,v_n\}$ and edge set $\{e_1,\ldots,e_m\}$.  An \emph{incidence matrix} of $G$ is an $n\times m$ matrix $N(G)$ whose $j$-th column is a the vector in $\mathbb{R}^n$ with $1$ and $-1$ on the entry corresponding to the two endpoints of $e_j$.  Thus, $L(G) = N(G)N(G)\trans$.

\section*{Problems}
\begin{enumerate}
\setlength\itemsep{2em}
\item Pick a graph and find its Laplacian matrix and one of its incidence matrix.
\item Pick a graph on $3$ vertices and check if the quadratic form of its Laplacian matrix follows the formula.
\item Pick a connected graph $G$ on $4$ vertices and let $L$ be its Laplacian matrix.  If $\bx$ is an eigenvector of $L$ with respect to $0$, what is $\bx$?
\item If $G$ and $H$ are two graphs such that $V(G) = V(H)$ and $E(G) \cap E(H) = \emptyset$, show that $L(G + H) = L(G) + L(H)$, where $G + H$ is the graph with vertex set $V(G)$ and edge set $E(G) \cup E(H)$.
\item Let $G$ be a graph and $L$ its Laplacian matrix.  For each edge $e\in E(G)$, define $H$ with $V(H_e) = V(G)$ and $E(H_e) = \{e\}$.  Thus, $G = \sum_{e\in E(G)} H_e$.  Use this fact to prove the formula of the quadratic form.
\end{enumerate}

% \newpage
% \section*{Questions to ponder}
% \begin{enumerate}
% \item 
% \end{enumerate}

\end{document}
