\documentclass{article}

%%%PACKAGE
\usepackage{amsmath,amssymb}
\usepackage{amsthm}
%% \usepackage[pdftex,bookmarks=true]{hyperref}
\usepackage{cite}
%% \usepackage{enumerate}
% \usepackage{url}
% \usepackage{hyperref}
% \usepackage{systeme}
%\usepackage{siunitx}
% \usepackage{multicol}
% \usepackage{systeme}

% for drawing graphs
\usepackage{tikz}
% \tikzset{every picture/.style={thick}}
\tikzset{every node/.style={draw, circle, inner sep = 2pt}}
\usetikzlibrary{arrows}

% for margins
\usepackage[margin=1in]{geometry}

% for font
\usepackage{euler}
\usepackage[OT1]{eulervm}
\renewcommand{\rmdefault}{pplx}

% \setlength{\parindent}{0pt}  %no indenting

% MACROS
\newcommand{\trans}{^\top}
\newcommand{\adj}{^{\rm adj}}
\newcommand{\cof}{^{\rm cof}}
\newcommand{\inp}[2]{\left\langle#1,#2\right\rangle}
\newcommand{\dunion}{\mathbin{\dot\cup}}
\newcommand{\bzero}{\mathbf{0}}
\newcommand{\bone}{\mathbf{1}}
\newcommand{\ba}{\mathbf{a}}
\newcommand{\bb}{\mathbf{b}}
\newcommand{\be}{\mathbf{e}}
\newcommand{\bp}{\mathbf{p}}
\newcommand{\bq}{\mathbf{q}}
\newcommand{\bx}{\mathbf{x}}
\newcommand{\by}{\mathbf{y}}
\newcommand{\bz}{\mathbf{z}}
\newcommand{\bu}{\mathbf{u}}
\newcommand{\bv}{\mathbf{v}}
\newcommand{\bw}{\mathbf{w}}
\newcommand{\tr}{\operatorname{tr}}
\newcommand{\nul}{\operatorname{null}}
\newcommand{\rank}{\operatorname{rank}}
%\newcommand{\ker}{\operatorname{ker}}
\newcommand{\range}{\operatorname{range}}
\newcommand{\Col}{\operatorname{Col}}
\newcommand{\Row}{\operatorname{Row}}
\newcommand{\spec}{\operatorname{spec}}
\newcommand{\vspan}{\operatorname{span}}
% \newenvironment{sol}{\medskip\noindent {\bf Solution.}}{\newpage}
\newcommand{\mystrut}{\rule[-.5\baselineskip]{0pt}{2\baselineskip}}
% \newcommand{\mul}{\operatorname{mul}}
\newcommand{\even}{\operatorname{even}}
\newcommand{\sgn}{\operatorname{sgn}}
\newcommand{\iner}{\operatorname{iner}}

%%%COMMENT
\usepackage{soul}
\usepackage{cancel}
\newcommand{\rbf}[1]{\textbf{\color{red}#1}}

%%%THEOREM
\newtheorem{theorem}{Theorem}[section]
\newtheorem{lemma}[theorem]{Lemma}
\newtheorem{proposition}[theorem]{Proposition}
\newtheorem{corollary}[theorem]{Corollary}

\theoremstyle{definition}
\newtheorem{definition}[theorem]{Definition}
\newtheorem{observation}[theorem]{Observation}
\newtheorem{remark}[theorem]{Remark}
\newtheorem{example}[theorem]{Example}
\newtheorem{notation}[theorem]{Notation}
\newtheorem{question}[theorem]{Question}

% for title
\title{Algebraic connectivity}
\date{\vspace{-1cm}}
\begin{document}
\maketitle
\large

Let $G$ be a simple graph.  The \emph{connectivity} of $G$ is the minimum number of vertices required such that by removing them the remaining graph is disconnected, $\kappa(G)$.  We vacuously define $\kappa(K_n) = n-1$.  

Let $G$ be a simple graph on $n$ vertices and $L$ its Laplacian matrix.  Let $\lambda_1 \leq \lambda_2 \leq \cdots \leq \lambda_n$ be the eigenvalues of $L$.  Recall that $\lambda_1 = 0$, and the multiplicity of $0$ is the number of components of $G$.  Equivalently, $\lambda_2 \neq 0$ if and only if $G$ is connected.  Moreover, it is known that $\lambda_2 \leq \kappa(G)$.  Therefore, we usually refer $\lambda_2$ as the \emph{algebraic connectivity} of $G$, denoted by $\lambda_2(G)$.  

The sketch of the proof of $\lambda_2(G) \leq \kappa(G)$ is as follows:
\begin{enumerate}
\item Let $S$ be a set such that $G\setminus S$ is disconnected and $|S| = \kappa(G)$.  Say $V(G\setminus S)$ can be partitioned into $X_1\dunion X_2$ such that there is no edge between them.  Let $n_1 = |X_1|$ and $n_2 = |X_2|$.  
\item If $e$ is not an edge of $G$, then $\lambda_2(G) \leq \lambda_2(G+e)$. (why?) Thus, we consider a new graph $H$ obtained from $G$ by adding all possible edges between $S$ and $V(G\setminus S)$.  Thus, 
\[\lambda_2(G) \leq \lambda_2(H).\]
\item Pick a vector $\bu$ in $\mathbb{R}^n$ that is $0$ on $S$, $n_2$ on $X_1$, and $-n_1$ on $X_2$.  Thus, $L(H)\bu = |S|\bu$, so $\lambda_2(H) \leq |S| = \kappa(G)$.  
\end{enumerate}

\section*{Problems}
\begin{enumerate}
\setlength\itemsep{2em}
\item Let $T$ be a tree.  What is $\kappa(T)$?    
\item What is the $\lambda_2(K_n)$?  
\item Pick a connected graph $G$.  Find $\lambda_2(G)$ and $\kappa(G)$. 
\item Prove $\lambda_2(G) \leq \lambda_2(G + e)$.  [Hint: Use Rayleigh quotient.]
\item Explain why $L(H)\bu = |S|\bu$.  
\item Use software if necessary, find a graph $G$ with $\kappa(G) - \lambda_2(G) \geq 1$.  
\end{enumerate}

% \newpage
% \section*{Questions to ponder}
% \begin{enumerate}
% \item 
% \end{enumerate}

\end{document}
