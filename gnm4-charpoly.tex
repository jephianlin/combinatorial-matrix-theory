\documentclass{article}

%%%PACKAGE
\usepackage{amsmath,amssymb}
\usepackage{amsthm}
%% \usepackage[pdftex,bookmarks=true]{hyperref}
\usepackage{cite}
%% \usepackage{enumerate}
% \usepackage{url}
% \usepackage{hyperref}
% \usepackage{systeme}
%\usepackage{siunitx}
% \usepackage{multicol}
% \usepackage{systeme}

% for drawing graphs
\usepackage{tikz}
% \tikzset{every picture/.style={thick}}
\tikzset{every node/.style={draw, circle, inner sep = 2pt}}
\usetikzlibrary{arrows}

% for margins
\usepackage[margin=1in]{geometry}

% for font
\usepackage{euler}
\usepackage[OT1]{eulervm}
\renewcommand{\rmdefault}{pplx}

% \setlength{\parindent}{0pt}  %no indenting

% MACROS
\newcommand{\trans}{^\top}
\newcommand{\adj}{^{\rm adj}}
\newcommand{\cof}{^{\rm cof}}
\newcommand{\inp}[2]{\left\langle#1,#2\right\rangle}
\newcommand{\dunion}{\mathbin{\dot\cup}}
\newcommand{\bzero}{\mathbf{0}}
\newcommand{\bone}{\mathbf{1}}
\newcommand{\ba}{\mathbf{a}}
\newcommand{\bb}{\mathbf{b}}
\newcommand{\bp}{\mathbf{p}}
\newcommand{\bq}{\mathbf{q}}
\newcommand{\bx}{\mathbf{x}}
\newcommand{\by}{\mathbf{y}}
\newcommand{\bz}{\mathbf{z}}
\newcommand{\bu}{\mathbf{u}}
\newcommand{\bv}{\mathbf{v}}
\newcommand{\bw}{\mathbf{w}}
\newcommand{\tr}{\operatorname{tr}}
\newcommand{\nul}{\operatorname{null}}
\newcommand{\rank}{\operatorname{rank}}
%\newcommand{\ker}{\operatorname{ker}}
\newcommand{\range}{\operatorname{range}}
\newcommand{\Col}{\operatorname{Col}}
\newcommand{\Row}{\operatorname{Row}}
\newcommand{\spec}{\operatorname{spec}}
\newcommand{\vspan}{\operatorname{span}}
% \newenvironment{sol}{\medskip\noindent {\bf Solution.}}{\newpage}
\newcommand{\mystrut}{\rule[-.5\baselineskip]{0pt}{2\baselineskip}}
% \newcommand{\mul}{\operatorname{mul}}
\newcommand{\even}{\operatorname{even}}
\newcommand{\sgn}{\operatorname{sgn}}

%%%COMMENT
\usepackage{soul}
\usepackage{cancel}
\newcommand{\rbf}[1]{\textbf{\color{red}#1}}

%%%THEOREM
\newtheorem{theorem}{Theorem}[section]
\newtheorem{lemma}[theorem]{Lemma}
\newtheorem{proposition}[theorem]{Proposition}
\newtheorem{corollary}[theorem]{Corollary}

\theoremstyle{definition}
\newtheorem{definition}[theorem]{Definition}
\newtheorem{observation}[theorem]{Observation}
\newtheorem{remark}[theorem]{Remark}
\newtheorem{example}[theorem]{Example}
\newtheorem{notation}[theorem]{Notation}
\newtheorem{question}[theorem]{Question}

% for title
\title{Characteristic polynomial}
\date{\vspace{-1cm}}
\begin{document}
\maketitle
\large

Let $A = \begin{bmatrix} a_{ij} \end{bmatrix}$ be an $n\times n$ matrix.  The \emph{characteristic polynomial} of $A$ is defined as 
\[\det(A - xI) = S_0(-x)^n + S_1(-x)^{n-1} + \cdots + S_n.\]
It is well-known that $S_0 = 1$, $S_1 = \tr(A)$, and $S_n = \det(A)$.  The other coefficients can also be written as the sum of principal minors.

Let $\alpha\subset[n]$.  Then $A[\alpha]$ is the submatrix induced on the rows in $\alpha$ and columns in $\alpha$.  Such a matrix is called a \emph{principal submatrix} and its determinant is called a \emph{principal minor}.  It is known that
\[S_k = \sum_{\substack{\alpha\subseteq [n]\\|\alpha| = k}} \det(A[\alpha]).\]

Let $\Gamma_A$ be the digraph of $A$.  Then $\alpha$ can be viewed as a subset of vertices of $\Gamma_A$.  An \emph{elementary subgraph of $\Gamma_A$ on $\alpha$} is a subgraph on $\alpha$ such that each vertex has in-degree $1$ and out-degree $1$ (equivalently, each component is either a loop, a doubly directed edges, or a cycle).  Let $H$ be an elementary subgraph of $\Gamma_A$.  Define $c(H)$ as the number of components of $H$.  The \emph{sign} of $H$ is $\sgn(H) = (-1)^{|\alpha|+c(H)}$.  The weight of $H$ is the product of the weights of its edges, denoted by $w(H)$.  
The set of all elementary subgraphs is denoted by $\mathfrak{E}_\alpha$, while $\mathfrak{E}_k = \bigcup_{\substack{\alpha\subseteq [n]\\|\alpha| = k}}\mathfrak{E}_\alpha$.  
Consequently, 
\[S_k = \sum_{H\in\mathfrak{E}_k} \sgn(H)w(H) = \sum_{H\in\mathfrak{E}_k}(-1)^{k+c(H)}w(H).\]

\section*{Problems}
\begin{enumerate}
\setlength\itemsep{2em}
\item Write a $4\times 4$ matrix.  Calculate $S_0,\ldots, S_4$ by its principal minors.
\item Use the same $A$ you had, draw the digraph and calculate $S_0,\ldots, S_4$ by the digraph.
\item Calculate the characteristic polynomial of  
\[A = \begin{bmatrix} 
 0 & 1 & 0 & 0 & 0 & 0 \\
 1 & 0 & 1 & 0 & 0 & 0 \\
 0 & 1 & 0 & 1 & 0 & 0 \\
 0 & 0 & 1 & 0 & 1 & 0 \\
 0 & 0 & 0 & 1 & 0 & 1 \\
 0 & 0 & 0 & 0 & 1 & 0 \\
\end{bmatrix}\] 
by the digraph.
\item When the matrix $A$ is symmetric and $\alpha\subseteq[n]$, how would you define the elementary subgraph on $\alpha$ of the simple graph of $A$?  How to calculate $S_k$?  What if $A$ is a symmetric matrix with zero diagonal?
\end{enumerate}

% \newpage
% \section*{Questions to ponder}
% \begin{enumerate}
% \item 
% \end{enumerate}

\end{document}
