\documentclass{article}

%%%PACKAGE
\usepackage{amsmath,amssymb}
\usepackage{amsthm}
%% \usepackage[pdftex,bookmarks=true]{hyperref}
\usepackage{cite}
%% \usepackage{enumerate}
% \usepackage{url}
% \usepackage{hyperref}
% \usepackage{systeme}
%\usepackage{siunitx}
% \usepackage{multicol}
% \usepackage{systeme}

% for drawing graphs
\usepackage{tikz}
% \tikzset{every picture/.style={thick}}
\tikzset{every node/.style={draw, circle, inner sep = 2pt}}
\usetikzlibrary{arrows}

% for margins
\usepackage[margin=1in]{geometry}

% for font
\usepackage{euler}
\usepackage[OT1]{eulervm}
\renewcommand{\rmdefault}{pplx}

% \setlength{\parindent}{0pt}  %no indenting

% MACROS
\newcommand{\trans}{^\top}
\newcommand{\adj}{^{\rm adj}}
\newcommand{\cof}{^{\rm cof}}
\newcommand{\inp}[2]{\left\langle#1,#2\right\rangle}
\newcommand{\dunion}{\mathbin{\dot\cup}}
\newcommand{\bzero}{\mathbf{0}}
\newcommand{\bone}{\mathbf{1}}
\newcommand{\ba}{\mathbf{a}}
\newcommand{\bb}{\mathbf{b}}
\newcommand{\bp}{\mathbf{p}}
\newcommand{\bq}{\mathbf{q}}
\newcommand{\bx}{\mathbf{x}}
\newcommand{\by}{\mathbf{y}}
\newcommand{\bz}{\mathbf{z}}
\newcommand{\bu}{\mathbf{u}}
\newcommand{\bv}{\mathbf{v}}
\newcommand{\bw}{\mathbf{w}}
\newcommand{\tr}{\operatorname{tr}}
\newcommand{\nul}{\operatorname{null}}
\newcommand{\rank}{\operatorname{rank}}
%\newcommand{\ker}{\operatorname{ker}}
\newcommand{\range}{\operatorname{range}}
\newcommand{\Col}{\operatorname{Col}}
\newcommand{\Row}{\operatorname{Row}}
\newcommand{\spec}{\operatorname{spec}}
\newcommand{\vspan}{\operatorname{span}}
% \newenvironment{sol}{\medskip\noindent {\bf Solution.}}{\newpage}
\newcommand{\mystrut}{\rule[-.5\baselineskip]{0pt}{2\baselineskip}}
% \newcommand{\mul}{\operatorname{mul}}
\newcommand{\even}{\operatorname{even}}
\newcommand{\sgn}{\operatorname{sgn}}
\newcommand{\iner}{\operatorname{iner}}

%%%COMMENT
\usepackage{soul}
\usepackage{cancel}
\newcommand{\rbf}[1]{\textbf{\color{red}#1}}

%%%THEOREM
\newtheorem{theorem}{Theorem}[section]
\newtheorem{lemma}[theorem]{Lemma}
\newtheorem{proposition}[theorem]{Proposition}
\newtheorem{corollary}[theorem]{Corollary}

\theoremstyle{definition}
\newtheorem{definition}[theorem]{Definition}
\newtheorem{observation}[theorem]{Observation}
\newtheorem{remark}[theorem]{Remark}
\newtheorem{example}[theorem]{Example}
\newtheorem{notation}[theorem]{Notation}
\newtheorem{question}[theorem]{Question}

% for title
\title{Interlacing}
\date{\vspace{-1cm}}
\begin{document}
\maketitle
\large

Let $A = \begin{bmatrix} a_{ij} \end{bmatrix}$ be an $n\times n$ real symmetric matrix.  Let $\lambda_1 \leq \cdots \leq \lambda_n$ be the eigenvalues of $A$.  Let $B$ be an $(n-1)\times (n-1)$ principal submatrix of $A$ and $\mu_1 \leq \cdots \leq \mu_{n-1}$ its eigenvalues.  Then 
\[\lambda_1 \leq \mu_1 \leq \lambda_2 \leq \cdots \leq \lambda_n.\]
More generally, if $B$ is an $k\times k$ principal submatrix of $A$ with eigenvalues $\mu_1 \leq \cdots \leq \mu_k$, then 
\[\lambda_i \leq \mu_i \leq \lambda_{n - (k-i)}.\]
  
\section*{Problems}
\begin{enumerate}
\setlength\itemsep{2em}
\item Pick a $3\times 3$ real symmetric matrix $A$ and one of its $2\times 2$ principal submatrix $B$.  Calculate their eigenvalues to witness the interlacing property.
\item Let $A$ be an $n\times n$ real symmetric matrix and $B$ one of its $(n-1)\times (n-1)$ principal submatrix.  If $\lambda$ is an eigenvalue of $A$ of multiplicity $m$, explain why $\lambda$ is an eigenvalue of multiplicity $m-1$, $m$, or $m+1$.  (When $\lambda$ is not an eigenvalue of $B$, we vaguely say its multiplicity is $0$.) 
\item Explain why $\lambda_1 \leq a_i \leq \lambda_n$ for each $i = 1,\ldots, n$ using the interlacing property.
\item Let $A$ be an $n\times n$ real symmetric matrix and $Q$ an $n\times n$ orthogonal matrix.  (That is, $QQ\trans = Q\trans Q = I$.)  Let $S$ be the $n\times k$ submatrix of $Q$ induced on the first $k$ columns.  Let $B = S\trans AS$ and $\mu_1 \leq \cdots \leq \mu_k$ its eigenvalues.  Explain why 
\[\lambda_i \leq \mu_i \leq \lambda_{n - (k-i)}.\]
\item Explain why $\lambda_1 \leq \frac{1}{n}\sum_{i=1}^n\sum_{j=1}^n a_{i,j} \leq \lambda_n$ using the interlacing property.
\end{enumerate}

% \newpage
% \section*{Questions to ponder}
% \begin{enumerate}
% \item 
% \end{enumerate}

\end{document}
