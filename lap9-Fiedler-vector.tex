\documentclass{article}

%%%PACKAGE
\usepackage{amsmath,amssymb}
\usepackage{amsthm}
%% \usepackage[pdftex,bookmarks=true]{hyperref}
\usepackage{cite}
%% \usepackage{enumerate}
% \usepackage{url}
% \usepackage{hyperref}
% \usepackage{systeme}
%\usepackage{siunitx}
% \usepackage{multicol}
% \usepackage{systeme}

% for drawing graphs
\usepackage{tikz}
% \tikzset{every picture/.style={thick}}
\tikzset{every node/.style={draw, circle, inner sep = 2pt}}
\usetikzlibrary{arrows}

% for margins
\usepackage[margin=1in]{geometry}

% for font
\usepackage{euler}
\usepackage[OT1]{eulervm}
\renewcommand{\rmdefault}{pplx}

% \setlength{\parindent}{0pt}  %no indenting

% MACROS
\newcommand{\trans}{^\top}
\newcommand{\adj}{^{\rm adj}}
\newcommand{\cof}{^{\rm cof}}
\newcommand{\inp}[2]{\left\langle#1,#2\right\rangle}
\newcommand{\dunion}{\mathbin{\dot\cup}}
\newcommand{\bzero}{\mathbf{0}}
\newcommand{\bone}{\mathbf{1}}
\newcommand{\ba}{\mathbf{a}}
\newcommand{\bb}{\mathbf{b}}
\newcommand{\be}{\mathbf{e}}
\newcommand{\bp}{\mathbf{p}}
\newcommand{\bq}{\mathbf{q}}
\newcommand{\bx}{\mathbf{x}}
\newcommand{\by}{\mathbf{y}}
\newcommand{\bz}{\mathbf{z}}
\newcommand{\bu}{\mathbf{u}}
\newcommand{\bv}{\mathbf{v}}
\newcommand{\bw}{\mathbf{w}}
\newcommand{\tr}{\operatorname{tr}}
\newcommand{\nul}{\operatorname{null}}
\newcommand{\rank}{\operatorname{rank}}
%\newcommand{\ker}{\operatorname{ker}}
\newcommand{\range}{\operatorname{range}}
\newcommand{\Col}{\operatorname{Col}}
\newcommand{\Row}{\operatorname{Row}}
\newcommand{\spec}{\operatorname{spec}}
\newcommand{\vspan}{\operatorname{span}}
% \newenvironment{sol}{\medskip\noindent {\bf Solution.}}{\newpage}
\newcommand{\mystrut}{\rule[-.5\baselineskip]{0pt}{2\baselineskip}}
% \newcommand{\mul}{\operatorname{mul}}
\newcommand{\even}{\operatorname{even}}
\newcommand{\sgn}{\operatorname{sgn}}
\newcommand{\iner}{\operatorname{iner}}
\newcommand{\rL}{\mathring{L}}

%%%COMMENT
\usepackage{soul}
\usepackage{cancel}
\newcommand{\rbf}[1]{\textbf{\color{red}#1}}

%%%THEOREM
\newtheorem{theorem}{Theorem}[section]
\newtheorem{lemma}[theorem]{Lemma}
\newtheorem{proposition}[theorem]{Proposition}
\newtheorem{corollary}[theorem]{Corollary}

\theoremstyle{definition}
\newtheorem{definition}[theorem]{Definition}
\newtheorem{observation}[theorem]{Observation}
\newtheorem{remark}[theorem]{Remark}
\newtheorem{example}[theorem]{Example}
\newtheorem{notation}[theorem]{Notation}
\newtheorem{question}[theorem]{Question}

% for title
\title{Fiedler vector}
\date{\vspace{-1cm}}
\begin{document}
\maketitle
\large

We have seen that the eigenvector corresponding to the second smallest Laplacian eigenvalue $\lambda_2$ (algebraic connectivity) is powerful on partitioning a graph.  This vector is called the \emph{Fiedler vector}, naming after the Czech mathematician Miroslav Fiedler.  

Let $T$ be a tree and $\bv = (v_1,\ldots,v_n)$ a Fiedler vector of $T$.  Then it is known that either one of the two cases will happen:
\begin{itemize}
\item There is a unique vertex $i$ such that $v_i = 0$ and it is adjacent to a vertex $j$ with $v_j \neq 0$.  
\item There is a unique pair of adjacenct vertices $i$ and $j$ such that $v_iv_j<0$.  
\end{itemize}
In fact, the set $\{i\}$ or $\{i,j\}$ coincide with the characteristic set found by the bottleneck matrices!

Moreover, the values on $\bv$ is monotone.  That is, for any path going from a vertex in the characteristic set to a leaf (without touching another vertex in the characteristic set), then the values of $\bv$ on this path is either monotone decreasing or monotone increasing.

More details on this topic can be found in \cite{BapatGM14}.

\begin{thebibliography}{9}
\bibitem{BapatGM14}
R.~B. Bapat.
\newblock {\em Graphs and Matrices}.
\newblock Springer-Verlag, London, 2nd edition, 2014.
\end{thebibliography}


\section*{Problems}
\begin{enumerate}
\setlength\itemsep{2em}
\item Let $T = P_4$.  Find a Fiedler vector.  (Essentially there is only one choice.)  Then use it to find the characteristic set.
\item Let $T = K_{1,3}$.  Find a Fiedler vector.  (There are many choices.)  Then use it to find the characteristic set.  Try a different Fiedler vector that is not parallel to the first one and see if the characteristic set is the same.
\item Prove that exactly one of the two cases mentioned above will happen.
\item Prove the monotonicity.  
\end{enumerate}

% \newpage
% \section*{Questions to ponder}
% \begin{enumerate}
% \item 
% \end{enumerate}

\end{document}
