\documentclass{article}

%%%PACKAGE
\usepackage{amsmath,amssymb}
\usepackage{amsthm}
%% \usepackage[pdftex,bookmarks=true]{hyperref}
\usepackage{cite}
%% \usepackage{enumerate}
% \usepackage{url}
% \usepackage{hyperref}
% \usepackage{systeme}
%\usepackage{siunitx}
% \usepackage{multicol}
% \usepackage{systeme}

% for drawing graphs
\usepackage{tikz}
% \tikzset{every picture/.style={thick}}
\tikzset{every node/.style={draw, circle, inner sep = 2pt}}
\usetikzlibrary{arrows}

% for margins
\usepackage[margin=1in]{geometry}

% for font
\usepackage{euler}
\usepackage[OT1]{eulervm}
\renewcommand{\rmdefault}{pplx}

% \setlength{\parindent}{0pt}  %no indenting

% MACROS
\newcommand{\trans}{^\top}
\newcommand{\adj}{^{\rm adj}}
\newcommand{\cof}{^{\rm cof}}
\newcommand{\inp}[2]{\left\langle#1,#2\right\rangle}
\newcommand{\dunion}{\mathbin{\dot\cup}}
\newcommand{\bzero}{\mathbf{0}}
\newcommand{\bone}{\mathbf{1}}
\newcommand{\ba}{\mathbf{a}}
\newcommand{\bb}{\mathbf{b}}
\newcommand{\bp}{\mathbf{p}}
\newcommand{\bq}{\mathbf{q}}
\newcommand{\bx}{\mathbf{x}}
\newcommand{\by}{\mathbf{y}}
\newcommand{\bz}{\mathbf{z}}
\newcommand{\bu}{\mathbf{u}}
\newcommand{\bv}{\mathbf{v}}
\newcommand{\bw}{\mathbf{w}}
\newcommand{\tr}{\operatorname{tr}}
\newcommand{\nul}{\operatorname{null}}
\newcommand{\rank}{\operatorname{rank}}
%\newcommand{\ker}{\operatorname{ker}}
\newcommand{\range}{\operatorname{range}}
\newcommand{\Col}{\operatorname{Col}}
\newcommand{\Row}{\operatorname{Row}}
\newcommand{\spec}{\operatorname{spec}}
\newcommand{\vspan}{\operatorname{span}}
% \newenvironment{sol}{\medskip\noindent {\bf Solution.}}{\newpage}
\newcommand{\mystrut}{\rule[-.5\baselineskip]{0pt}{2\baselineskip}}
% \newcommand{\mul}{\operatorname{mul}}
\newcommand{\even}{\operatorname{even}}
\newcommand{\sgn}{\operatorname{sgn}}

%%%COMMENT
\usepackage{soul}
\usepackage{cancel}
\newcommand{\rbf}[1]{\textbf{\color{red}#1}}

%%%THEOREM
\newtheorem{theorem}{Theorem}[section]
\newtheorem{lemma}[theorem]{Lemma}
\newtheorem{proposition}[theorem]{Proposition}
\newtheorem{corollary}[theorem]{Corollary}

\theoremstyle{definition}
\newtheorem{definition}[theorem]{Definition}
\newtheorem{observation}[theorem]{Observation}
\newtheorem{remark}[theorem]{Remark}
\newtheorem{example}[theorem]{Example}
\newtheorem{notation}[theorem]{Notation}
\newtheorem{question}[theorem]{Question}

% for title
\title{Determinant}
\date{\vspace{-1cm}}
\begin{document}
\maketitle
\large

A \emph{permutation} on $[n]$ is a bijection on $[n]$.  Let $\sigma$ be a permutation on $[n]$.  Define $c(\sigma)$ as the number of cycles of $\sigma$ in its cycle representation.  
The \emph{permutation matrix} of $\sigma$ is the $n\times n$ matrix $P_\sigma$ whose $i,\sigma(i)$-entry is $1$ for $i = 1,\ldots, n$ while others are $0$.  The \emph{sign} of $\sigma$ defined as $\sgn(\sigma) = \det(P_\sigma)$.  It is known that $\sgn(\sigma) = (-1)^{n + c(\sigma)}$.  The set (group) of all permutations on $[n]$ is denoted by $\mathfrak{S}_n$.

Let $A = \begin{bmatrix} a_{ij} \end{bmatrix}$ be an $n\times n$ matrix.  Then 
\[\det(A) = 
\sum_{\sigma\in\mathfrak{S}_n} \sgn(\sigma)a_{1\sigma(1)}\cdots a_{n\sigma(n)} = 
\sum_{\sigma\in\mathfrak{S}_n} (-1)^{n+c(\sigma)}w(\sigma),\]
where $w(\sigma) = a_{1\sigma(1)}\cdots a_{n\sigma(n)}$ can be viewed as the weight of $\sigma$ defined by $A$.

Let $\Gamma_A$ be the weighted digraph of $A$.  An \emph{elementary subgraph} of $\Gamma_A$ is a spanning subgraph of $\Gamma_A$ such that each vertex has in-degree $1$ and out-degree $1$ (equivalently, each component is either a loop, a doubly directed edges, or a cycle).  Let $H$ be an elementary subgraph of $\Gamma_A$.  Define $c(H)$ as the number of components of $H$.  The \emph{sign} of $H$ is $\sgn(H) = (-1)^{n+c(H)}$.  The weight of $H$ is the product of the weights of its edges, denoted by $w(H)$.  
The set of all elementary subgraphs is denoted by $\mathfrak{E}$.  Consequently, 
\[\det(A) = \sum_{H\in\mathfrak{E}} \sgn(H)w(H) = \sum_{H\in\mathfrak{E}}(-1)^{n+c(H)}w(H).\]

\section*{Problems}
\begin{enumerate}
\setlength\itemsep{2em}
\item Write a $4\times 4$ matrix.  List all elements in $\mathfrak{S}_4$ and calculate the determinant by the permutation expansion.
\item Use the same $A$ you had, draw the digraph, list all graphs in $\mathfrak{E}$, and calculate the determinant by the digraph.
\item Calculate the determinant of  
\[A = \begin{bmatrix} 
 0 & 1 & 0 & 0 & 0 & 0 \\
 1 & 0 & 1 & 0 & 0 & 0 \\
 0 & 1 & 0 & 1 & 0 & 0 \\
 0 & 0 & 1 & 0 & 1 & 0 \\
 0 & 0 & 0 & 1 & 0 & 1 \\
 0 & 0 & 0 & 0 & 1 & 0 \\
\end{bmatrix}\] 
by the digraph.
\item When the matrix $A$ is symmetric, how would you define the elementary subgraph of the simple graph of $A$?  How to calculate $\det(A)$?  What if $A$ is a symmetric matrix with zero diagonal?
\end{enumerate}

% \newpage
% \section*{Questions to ponder}
% \begin{enumerate}
% \item 
% \end{enumerate}

\end{document}
