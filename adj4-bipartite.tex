\documentclass{article}

%%%PACKAGE
\usepackage{amsmath,amssymb}
\usepackage{amsthm}
%% \usepackage[pdftex,bookmarks=true]{hyperref}
\usepackage{cite}
%% \usepackage{enumerate}
% \usepackage{url}
% \usepackage{hyperref}
% \usepackage{systeme}
%\usepackage{siunitx}
% \usepackage{multicol}
% \usepackage{systeme}

% for drawing graphs
\usepackage{tikz}
% \tikzset{every picture/.style={thick}}
\tikzset{every node/.style={draw, circle, inner sep = 2pt}}
\usetikzlibrary{arrows}

% for margins
\usepackage[margin=1in]{geometry}

% for font
\usepackage{euler}
\usepackage[OT1]{eulervm}
\renewcommand{\rmdefault}{pplx}

% \setlength{\parindent}{0pt}  %no indenting

% MACROS
\newcommand{\trans}{^\top}
\newcommand{\adj}{^{\rm adj}}
\newcommand{\cof}{^{\rm cof}}
\newcommand{\inp}[2]{\left\langle#1,#2\right\rangle}
\newcommand{\dunion}{\mathbin{\dot\cup}}
\newcommand{\bzero}{\mathbf{0}}
\newcommand{\bone}{\mathbf{1}}
\newcommand{\ba}{\mathbf{a}}
\newcommand{\bb}{\mathbf{b}}
\newcommand{\bp}{\mathbf{p}}
\newcommand{\bq}{\mathbf{q}}
\newcommand{\bx}{\mathbf{x}}
\newcommand{\by}{\mathbf{y}}
\newcommand{\bz}{\mathbf{z}}
\newcommand{\bu}{\mathbf{u}}
\newcommand{\bv}{\mathbf{v}}
\newcommand{\bw}{\mathbf{w}}
\newcommand{\tr}{\operatorname{tr}}
\newcommand{\nul}{\operatorname{null}}
\newcommand{\rank}{\operatorname{rank}}
%\newcommand{\ker}{\operatorname{ker}}
\newcommand{\range}{\operatorname{range}}
\newcommand{\Col}{\operatorname{Col}}
\newcommand{\Row}{\operatorname{Row}}
\newcommand{\spec}{\operatorname{spec}}
\newcommand{\vspan}{\operatorname{span}}
% \newenvironment{sol}{\medskip\noindent {\bf Solution.}}{\newpage}
\newcommand{\mystrut}{\rule[-.5\baselineskip]{0pt}{2\baselineskip}}
% \newcommand{\mul}{\operatorname{mul}}
\newcommand{\even}{\operatorname{even}}
\newcommand{\sgn}{\operatorname{sgn}}
\newcommand{\iner}{\operatorname{iner}}

%%%COMMENT
\usepackage{soul}
\usepackage{cancel}
\newcommand{\rbf}[1]{\textbf{\color{red}#1}}

%%%THEOREM
\newtheorem{theorem}{Theorem}[section]
\newtheorem{lemma}[theorem]{Lemma}
\newtheorem{proposition}[theorem]{Proposition}
\newtheorem{corollary}[theorem]{Corollary}

\theoremstyle{definition}
\newtheorem{definition}[theorem]{Definition}
\newtheorem{observation}[theorem]{Observation}
\newtheorem{remark}[theorem]{Remark}
\newtheorem{example}[theorem]{Example}
\newtheorem{notation}[theorem]{Notation}
\newtheorem{question}[theorem]{Question}

% for title
\title{Bipartite graphs}
\date{\vspace{-1cm}}
\begin{document}
\maketitle
\large

Recall that a graph $G$ is a bipartite graph if $V(G)$ can be partitioned as $X\dunion Y$ such that all edges are between $X$ and $Y$.  Equivalently, $G$ is a bipartite graph if and only if the adjacency matrix of $G$ can be written as 
\[A = \begin{bmatrix}
 O_{|X|} & B\trans \\
 B & O_{|Y|}
\end{bmatrix}.\]
Suppose $A$ has an eigenvector $\bv$ with respect to the eigenvalue $\lambda$.  Then the equation $A\bv = \lambda\bv$ is equivalent to 
\[\begin{aligned}
B\bx &= \lambda\by, \\
B\trans\by &= \lambda\bx,
\end{aligned} \text{ where } 
\bv = \begin{bmatrix} \bx \\ \by \end{bmatrix}.\]
Direct computation leads to the relation that 
\[A\begin{bmatrix} \bx \\ \by \end{bmatrix} = \lambda\begin{bmatrix} \bx \\ \by \end{bmatrix} \iff 
A\begin{bmatrix} \bx \\ -\by \end{bmatrix} = -\lambda\begin{bmatrix} \bx \\ -\by \end{bmatrix}.\]
Therefore, if $\lambda$ is an eigenvalue of $A$, then $-\lambda$ is also an eigenvalue of $A$ of the same multiplicity.  

In fact, if a graph $G$ with adjacency matrix $A$ has its adjacency eigenvalues symmetric to $0$, then we know $G$ is a bipartite graph for the following reasons.  Since the eigenvalues are symmetric to $0$, we know the adjacency characteristic polynomial of $G$ has the form 
\[p(x) = (-x)^k\prod_{\lambda}(\lambda - x)(-\lambda - x) = (-x)^k\prod_{\lambda}(\lambda^2 - x^2),\]
where $k$ is the nullity of $A$ and $\lambda$ runs through all positive eigenvalues of $A$, counting the multiplicity.  Therefore, $p(x) / x^k$ is a even function,  which means $S_1 = S_3 = \cdots = 0$.  Since $S_3 = 0$, $G$ contains no triangle $C_3$.  Since $G$ contains no $C_3$ and $S_5 = 0$, $G$ contains no $C_5$.  Inductively, we know $G$ contains no odd cycles.  It is well-known that if $G$ contains no odd cycles, then $G$ is a bipartite graph.

\section*{Problems}
\begin{enumerate}
\setlength\itemsep{2em}
\item Let $G = K_{1,3}$ and $A$ its adjacency matrix.  Find the characteristic polynomial, the eigenvalues, an eigenbasis of $A$.  
\item Let $G = K_{2,2}$ and $A$ its adjacency matrix.  Find the characteristic polynomial, the eigenvalues, an eigenbasis of $A$.  
\item Explain why the adjacency matrix of a bipartite graph on odd number of vertices is always singular.
\item Find the eigenvalue and an eigenbasis for the adjacency matrix of a complete graph $K_{m,n}$.  
\item Let $k$ be the smallest positive odd integer such that $S_k \neq 0$.  Prove that the shortest odd cycle on $G$ has length $k$.
\end{enumerate}

% \newpage
% \section*{Questions to ponder}
% \begin{enumerate}
% \item 
% \end{enumerate}

\end{document}
