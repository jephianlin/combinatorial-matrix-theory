\documentclass{article}

%%%PACKAGE
\usepackage{amsmath,amssymb}
\usepackage{amsthm}
%% \usepackage[pdftex,bookmarks=true]{hyperref}
\usepackage{cite}
%% \usepackage{enumerate}
% \usepackage{url}
% \usepackage{hyperref}
% \usepackage{systeme}
%\usepackage{siunitx}
% \usepackage{multicol}
% \usepackage{systeme}

% for drawing graphs
\usepackage{tikz}
% \tikzset{every picture/.style={thick}}
\tikzset{every node/.style={draw, circle, inner sep = 2pt}}
\usetikzlibrary{arrows}

% for margins
\usepackage[margin=1in]{geometry}

% for font
\usepackage{euler}
\usepackage[OT1]{eulervm}
\renewcommand{\rmdefault}{pplx}

% \setlength{\parindent}{0pt}  %no indenting

% MACROS
\newcommand{\trans}{^\top}
\newcommand{\adj}{^{\rm adj}}
\newcommand{\cof}{^{\rm cof}}
\newcommand{\inp}[2]{\left\langle#1,#2\right\rangle}
\newcommand{\dunion}{\mathbin{\dot\cup}}
\newcommand{\bzero}{\mathbf{0}}
\newcommand{\bone}{\mathbf{1}}
\newcommand{\ba}{\mathbf{a}}
\newcommand{\bb}{\mathbf{b}}
\newcommand{\bp}{\mathbf{p}}
\newcommand{\bq}{\mathbf{q}}
\newcommand{\bx}{\mathbf{x}}
\newcommand{\by}{\mathbf{y}}
\newcommand{\bz}{\mathbf{z}}
\newcommand{\bu}{\mathbf{u}}
\newcommand{\bv}{\mathbf{v}}
\newcommand{\bw}{\mathbf{w}}
\newcommand{\tr}{\operatorname{tr}}
\newcommand{\nul}{\operatorname{null}}
\newcommand{\rank}{\operatorname{rank}}
%\newcommand{\ker}{\operatorname{ker}}
\newcommand{\range}{\operatorname{range}}
\newcommand{\Col}{\operatorname{Col}}
\newcommand{\Row}{\operatorname{Row}}
\newcommand{\spec}{\operatorname{spec}}
\newcommand{\vspan}{\operatorname{span}}
% \newenvironment{sol}{\medskip\noindent {\bf Solution.}}{\newpage}
\newcommand{\mystrut}{\rule[-.5\baselineskip]{0pt}{2\baselineskip}}
% \newcommand{\mul}{\operatorname{mul}}
\newcommand{\even}{\operatorname{even}}
\newcommand{\sgn}{\operatorname{sgn}}
\newcommand{\iner}{\operatorname{iner}}

%%%COMMENT
\usepackage{soul}
\usepackage{cancel}
\newcommand{\rbf}[1]{\textbf{\color{red}#1}}

%%%THEOREM
\newtheorem{theorem}{Theorem}[section]
\newtheorem{lemma}[theorem]{Lemma}
\newtheorem{proposition}[theorem]{Proposition}
\newtheorem{corollary}[theorem]{Corollary}

\theoremstyle{definition}
\newtheorem{definition}[theorem]{Definition}
\newtheorem{observation}[theorem]{Observation}
\newtheorem{remark}[theorem]{Remark}
\newtheorem{example}[theorem]{Example}
\newtheorem{notation}[theorem]{Notation}
\newtheorem{question}[theorem]{Question}

% for title
\title{Component}
\date{\vspace{-1cm}}
\begin{document}
\maketitle
\large

Let $G$ be a connected graph on $n$ vertices and $L$ its Laplacian matrix.  By the quadratic form, 
\[\nul(L) = 1 \text{ and }\ker(L) = \vspan\{\bone\}.\]
In general, if $G$ has $k$ components on the vertex sets $X_1,\ldots, X_k$, then 
\[\nul(L) = k \text{ and } \ker(L) = \vspan\{\phi_1,\ldots,\phi_k\},\] 
where $\phi_i$ is the characteristic vector of $X_i$ for $i = 1,\ldots, k$.

In fact, the same idea works for weighted graph.  Let $G$ be a weighted graph such that $w(i,j) = w(j,i) > 0$ is the weight for the edge $\{i,j\}\in E(G)$.  We may defined the \emph{weighted Laplacian matrix} $L(G) = \begin{bmatrix} \ell_{ij} \end{bmatrix}$ such that 
\[\ell_{ij} = \begin{cases}
\sum_{k:k\sim i} w(i,k) & \text{if }i = j, \\
-w(i,j) & \text{if }\{i,j\}\in E(G), \\
0 & otherwise.
\end{cases}\]
Thus, 
\[\bx\trans L\bx = \sum_{\{i,j\}\in E(G)}w(i,j)(x_i - x_j)^2,\]
and the same properties regarding the components hold.

\section*{Problems}
\begin{enumerate}
\setlength\itemsep{2em}
\item Pick a graph with $3$ components and let $L$ be its Laplacian matrix.  Find $\nul(L)$ and $\ker(L)$.
\item If $G$ is connected and $L$ is its Laplacian matrix, prove that $\ker(L) = \vspan\{\bone\}$.  
\item Let $G$ be the graph obtained from two copies of $K_{10}$ by adding on edge between them.  Use software if necessary, find the eigenvalues of $L(G)$.  How many ``zeroish'' eigenvalues does $G$ have?  How would you guess the number of ``clusters'' of $G$?
\item Write down a connected weighted graph.  Find its weighted Laplacian matrix $L$ and check if $\nul(L) = 1$ and $\ker(L) = \vspan\{\bone\}$.
\end{enumerate}

% \newpage
% \section*{Questions to ponder}
% \begin{enumerate}
% \item 
% \end{enumerate}

\end{document}
