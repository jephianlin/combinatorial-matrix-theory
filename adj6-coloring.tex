\documentclass{article}

%%%PACKAGE
\usepackage{amsmath,amssymb}
\usepackage{amsthm}
%% \usepackage[pdftex,bookmarks=true]{hyperref}
\usepackage{cite}
%% \usepackage{enumerate}
% \usepackage{url}
% \usepackage{hyperref}
% \usepackage{systeme}
%\usepackage{siunitx}
% \usepackage{multicol}
% \usepackage{systeme}

% for drawing graphs
\usepackage{tikz}
% \tikzset{every picture/.style={thick}}
\tikzset{every node/.style={draw, circle, inner sep = 2pt}}
\usetikzlibrary{arrows}

% for margins
\usepackage[margin=1in]{geometry}

% for font
\usepackage{euler}
\usepackage[OT1]{eulervm}
\renewcommand{\rmdefault}{pplx}

% \setlength{\parindent}{0pt}  %no indenting

% MACROS
\newcommand{\trans}{^\top}
\newcommand{\adj}{^{\rm adj}}
\newcommand{\cof}{^{\rm cof}}
\newcommand{\inp}[2]{\left\langle#1,#2\right\rangle}
\newcommand{\dunion}{\mathbin{\dot\cup}}
\newcommand{\bzero}{\mathbf{0}}
\newcommand{\bone}{\mathbf{1}}
\newcommand{\ba}{\mathbf{a}}
\newcommand{\bb}{\mathbf{b}}
\newcommand{\bp}{\mathbf{p}}
\newcommand{\bq}{\mathbf{q}}
\newcommand{\bx}{\mathbf{x}}
\newcommand{\by}{\mathbf{y}}
\newcommand{\bz}{\mathbf{z}}
\newcommand{\bu}{\mathbf{u}}
\newcommand{\bv}{\mathbf{v}}
\newcommand{\bw}{\mathbf{w}}
\newcommand{\tr}{\operatorname{tr}}
\newcommand{\nul}{\operatorname{null}}
\newcommand{\rank}{\operatorname{rank}}
%\newcommand{\ker}{\operatorname{ker}}
\newcommand{\range}{\operatorname{range}}
\newcommand{\Col}{\operatorname{Col}}
\newcommand{\Row}{\operatorname{Row}}
\newcommand{\spec}{\operatorname{spec}}
\newcommand{\vspan}{\operatorname{span}}
% \newenvironment{sol}{\medskip\noindent {\bf Solution.}}{\newpage}
\newcommand{\mystrut}{\rule[-.5\baselineskip]{0pt}{2\baselineskip}}
% \newcommand{\mul}{\operatorname{mul}}
\newcommand{\even}{\operatorname{even}}
\newcommand{\sgn}{\operatorname{sgn}}
\newcommand{\iner}{\operatorname{iner}}

%%%COMMENT
\usepackage{soul}
\usepackage{cancel}
\newcommand{\rbf}[1]{\textbf{\color{red}#1}}

%%%THEOREM
\newtheorem{theorem}{Theorem}[section]
\newtheorem{lemma}[theorem]{Lemma}
\newtheorem{proposition}[theorem]{Proposition}
\newtheorem{corollary}[theorem]{Corollary}

\theoremstyle{definition}
\newtheorem{definition}[theorem]{Definition}
\newtheorem{observation}[theorem]{Observation}
\newtheorem{remark}[theorem]{Remark}
\newtheorem{example}[theorem]{Example}
\newtheorem{notation}[theorem]{Notation}
\newtheorem{question}[theorem]{Question}

% for title
\title{Chromatic number}
\date{\vspace{-1cm}}
\begin{document}
\maketitle
\large

Let $G$ be a graph on $n$ vertices and $A$ its adjacency matrix.  A $k$-proper coloring of $G$ is a partition of $V(G)$ into $X_1\dunion \cdots \dunion X_k$ such that each of $X_i$ is an independent set for $i = 1,\ldots, k$.  Equivalently, a partition $X_1\dunion \cdots \dunion X_k$ of $V(G)$ is a $k$-proper coloring if and only if $A$ can be written as 
\[\begin{bmatrix}
 O_{|X_1|} & ? & \cdots & ? \\
 ? & O_{|X_2|} & \ddots & \vdots \\
 \vdots & \ddots & \ddots & ? \\
 ? & \cdots & ? & O_{|X_k|} 
\end{bmatrix}.\]
The \emph{chromatic number} of $G$ is the minimum $k$ such that $G$ has a $k$-proper coloring, denoted by $\chi(G)$.

Let $G$ be a graph on $n$ vertices, $A$ its adjacency matrix, and $\lambda_1\leq \cdots \leq \lambda_n$ the eigenvalues of $A$.  The \textbf{Hoffman bound} states that 
\[\chi(G) \geq 1 - \frac{\lambda_n}{\lambda_1}\]
whenever $G$ contains at least an edge.  One of the proofs of the Hoffman bound utilizes the \emph{weight-regular partition}.  The sketch of the proof is as follows.
\begin{enumerate}
\item Suppose $\chi(G) = k$ and $X_1\dunion \cdots \dunion X_k$ is a $k$-proper coloring.  Let $\phi_1,\ldots,\phi_k$ be the corresponding characteristic vectors.
\item Let $\bu$ be an eigenvector corresponding to $\lambda_n$.  Define $\bx_i$ as the entrywise product of $\bu$ and $\phi_i$.  If $\bx_i = \bzero$, replace it with $\phi_i$.
\item Define an $k\times k$ matrix $C = \begin{bmatrix} c_{ij} \end{bmatrix}$ such that $c_{ij} = \frac{1}{\|\bx_i\|\|\bx_j\|}\bx_i\trans A\bx_j$.  Thus, the eigenvalues of $C$ interlace with the eigenvalues of $A$.  (Why?)  Let $\mu_1 \leq \cdots \leq \mu_k$ be the eigenvalues of $C$.
\item Since $\bu$ is in the span of $\{\bx_i\}_{i=1}^k$, $\mu_k = \lambda_n$.  
\item The diagonal entries of $C$ are zero, so 
\[\begin{aligned} 
 0 = \tr(C) &= \mu_1 + \cdots + \mu_k \\
 &\geq \lambda_1 + \cdots + \lambda_{k-1} + \lambda_n \\
 &\geq (k-1)\lambda_1 + \lambda_n.
\end{aligned}\]
Therefore, $\chi(G) \geq 1 - \frac{\lambda_n}{\lambda_1}$.  
\end{enumerate}

\section*{Problems}
\begin{enumerate}
\setlength\itemsep{2em}
\item Pick a graph, calculate its chromatic number and its eigenvalues, and then check if the Hoffman bound is tight.   
\item Check if the Hoffman bound is tight for complete graph $K_n$ with $n\geq 2$.  
\item Prove the Hoffman bound for bipartite graphs. 
\item Prove the statement in step 3.
\item Let $C$ be the matrix in the proof.  Find an eigenvector for $C$ with respect to $\lambda_n$.
\end{enumerate}

% \newpage
% \section*{Questions to ponder}
% \begin{enumerate}
% \item 
% \end{enumerate}

\end{document}
