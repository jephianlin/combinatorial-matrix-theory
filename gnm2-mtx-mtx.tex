\documentclass{article}

%%%PACKAGE
\usepackage{amsmath,amssymb}
\usepackage{amsthm}
%% \usepackage[pdftex,bookmarks=true]{hyperref}
\usepackage{cite}
%% \usepackage{enumerate}
% \usepackage{url}
% \usepackage{hyperref}
% \usepackage{systeme}
%\usepackage{siunitx}
% \usepackage{multicol}
% \usepackage{systeme}

% for drawing graphs
\usepackage{tikz}
% \tikzset{every picture/.style={thick}}
\tikzset{every node/.style={draw, circle, inner sep = 2pt}}
\usetikzlibrary{arrows}

% for margins
\usepackage[margin=1in]{geometry}

% for font
\usepackage{euler}
\usepackage[OT1]{eulervm}
\renewcommand{\rmdefault}{pplx}

% \setlength{\parindent}{0pt}  %no indenting

% MACROS
\newcommand{\trans}{^\top}
\newcommand{\adj}{^{\rm adj}}
\newcommand{\cof}{^{\rm cof}}
\newcommand{\inp}[2]{\left\langle#1,#2\right\rangle}
\newcommand{\dunion}{\mathbin{\dot\cup}}
\newcommand{\bzero}{\mathbf{0}}
\newcommand{\bone}{\mathbf{1}}
\newcommand{\ba}{\mathbf{a}}
\newcommand{\bb}{\mathbf{b}}
\newcommand{\bp}{\mathbf{p}}
\newcommand{\bq}{\mathbf{q}}
\newcommand{\bx}{\mathbf{x}}
\newcommand{\by}{\mathbf{y}}
\newcommand{\bz}{\mathbf{z}}
\newcommand{\bu}{\mathbf{u}}
\newcommand{\bv}{\mathbf{v}}
\newcommand{\bw}{\mathbf{w}}
\newcommand{\tr}{\operatorname{tr}}
\newcommand{\nul}{\operatorname{null}}
\newcommand{\rank}{\operatorname{rank}}
%\newcommand{\ker}{\operatorname{ker}}
\newcommand{\range}{\operatorname{range}}
\newcommand{\Col}{\operatorname{Col}}
\newcommand{\Row}{\operatorname{Row}}
\newcommand{\spec}{\operatorname{spec}}
\newcommand{\vspan}{\operatorname{span}}
% \newenvironment{sol}{\medskip\noindent {\bf Solution.}}{\newpage}
\newcommand{\mystrut}{\rule[-.5\baselineskip]{0pt}{2\baselineskip}}
% \newcommand{\mul}{\operatorname{mul}}

%%%COMMENT
\usepackage{soul}
\usepackage{cancel}
\newcommand{\rbf}[1]{\textbf{\color{red}#1}}

%%%THEOREM
\newtheorem{theorem}{Theorem}[section]
\newtheorem{lemma}[theorem]{Lemma}
\newtheorem{proposition}[theorem]{Proposition}
\newtheorem{corollary}[theorem]{Corollary}

\theoremstyle{definition}
\newtheorem{definition}[theorem]{Definition}
\newtheorem{observation}[theorem]{Observation}
\newtheorem{remark}[theorem]{Remark}
\newtheorem{example}[theorem]{Example}
\newtheorem{notation}[theorem]{Notation}
\newtheorem{question}[theorem]{Question}

% for title
\title{Matrix-matrix multiplication}
\date{\vspace{-1cm}}
\begin{document}
\maketitle
\large

Let $A = \begin{bmatrix} a_{ij} \end{bmatrix}$ and $B = \begin{bmatrix} b_{ij} \end{bmatrix}$ be two $n\times n$ matrices.  Then $AB$ is another $n\times n$ matrix whose $i,j$-entry is
\[\sum_{k\in[n]}a_{ik}b_{kj}.\]

Let $\Gamma_A$ and $\Gamma_B$ be the digraphs of $A$ and $B$, respectively.  Similar to the out-neighborhood, we define the \emph{in-neighborhood} of a vertex $j$ of a digraph $\Gamma$ as 
\[N^-_\Gamma(v) = \{k : (k,j) \in E(\Gamma)\}.\]
Thus, the $i,j$-entry of $AB$ is  
\[\sum_{k\in N^+_{\Gamma_A}(i)\cap N^-_{\Gamma_B}(j)}a_{ik}b_{kj}.\]
When $A = B$, the formula can be cleaner as follows.  
\[\sum_{\substack{k\\ i\rightarrow k\rightarrow j}}a_{ik}a_{kj}.\]

Moreover $\Gamma_A$ can be viewed as a weighted digraph, where $w(i,j) = a_{ij}$ is the weight on the edge $(i,j)$.  
A \emph{walk} $W$ on $\Gamma_A$ is a sequence of vertices and edges
\[v_0\rightarrow v_1\rightarrow \cdots \rightarrow v_r\]
such that each $v_k$ is a vertex for $k=0,\ldots,r$ and $(v_k,v_{k+1})\in E(\Gamma_A)$ for each $k = 0,\ldots, r-1$.  The \emph{length} of $W$ is the number of edges on it.  The \emph{weight} of $W$ is the product of the weights of its edges, denoted by $w(W)$.  Thus, the $i,j$-entry of $A^r$ is 
\[\sum_{W}w(W),\]
where $W$ runs through all walks from $i$ to $j$ of length $r$ on $\Gamma_A$. 
\section*{Problems}
\begin{enumerate}
\setlength\itemsep{2em}
\item Write two $4\times 4$ matrices $A$ and $B$, draw the digraphs of them, and calculate $AB$ by the digraphs.
\item Use the same $A$ you had, calculate $A^2$ and $A^3$.
\item Let 
\[A = \begin{bmatrix} 
 0 & 1 & 0 & 0 & 0 & 0 \\
 0 & 0 & 1 & 0 & 0 & 0 \\
 0 & 0 & 0 & 1 & 0 & 0 \\
 0 & 0 & 0 & 0 & 1 & 0 \\
 0 & 0 & 0 & 0 & 0 & 1 \\
 1 & 0 & 0 & 0 & 0 & 0 \\
\end{bmatrix}\] 
 be a permutation matrix.  Draw the digraphs of $A^2$, $A^3$, and $A^6$.
\item When the matrix $A$ is symmetric, how would you define a walk and its weight on the simple graph of $A$?  How to calculate the $i,j$-entry of $A^r$?  What if $A$ is a symmetric matrix with zero diagonal?
\end{enumerate}

% \newpage
% \section*{Questions to ponder}
% \begin{enumerate}
% \item 
% \end{enumerate}

\end{document}
