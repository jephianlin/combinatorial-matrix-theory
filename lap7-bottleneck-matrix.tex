\documentclass{article}

%%%PACKAGE
\usepackage{amsmath,amssymb}
\usepackage{amsthm}
%% \usepackage[pdftex,bookmarks=true]{hyperref}
\usepackage{cite}
%% \usepackage{enumerate}
% \usepackage{url}
% \usepackage{hyperref}
% \usepackage{systeme}
%\usepackage{siunitx}
% \usepackage{multicol}
% \usepackage{systeme}

% for drawing graphs
\usepackage{tikz}
% \tikzset{every picture/.style={thick}}
\tikzset{every node/.style={draw, circle, inner sep = 2pt}}
\usetikzlibrary{arrows}

% for margins
\usepackage[margin=1in]{geometry}

% for font
\usepackage{euler}
\usepackage[OT1]{eulervm}
\renewcommand{\rmdefault}{pplx}

% \setlength{\parindent}{0pt}  %no indenting

% MACROS
\newcommand{\trans}{^\top}
\newcommand{\adj}{^{\rm adj}}
\newcommand{\cof}{^{\rm cof}}
\newcommand{\inp}[2]{\left\langle#1,#2\right\rangle}
\newcommand{\dunion}{\mathbin{\dot\cup}}
\newcommand{\bzero}{\mathbf{0}}
\newcommand{\bone}{\mathbf{1}}
\newcommand{\ba}{\mathbf{a}}
\newcommand{\bb}{\mathbf{b}}
\newcommand{\be}{\mathbf{e}}
\newcommand{\bp}{\mathbf{p}}
\newcommand{\bq}{\mathbf{q}}
\newcommand{\bx}{\mathbf{x}}
\newcommand{\by}{\mathbf{y}}
\newcommand{\bz}{\mathbf{z}}
\newcommand{\bu}{\mathbf{u}}
\newcommand{\bv}{\mathbf{v}}
\newcommand{\bw}{\mathbf{w}}
\newcommand{\tr}{\operatorname{tr}}
\newcommand{\nul}{\operatorname{null}}
\newcommand{\rank}{\operatorname{rank}}
%\newcommand{\ker}{\operatorname{ker}}
\newcommand{\range}{\operatorname{range}}
\newcommand{\Col}{\operatorname{Col}}
\newcommand{\Row}{\operatorname{Row}}
\newcommand{\spec}{\operatorname{spec}}
\newcommand{\vspan}{\operatorname{span}}
% \newenvironment{sol}{\medskip\noindent {\bf Solution.}}{\newpage}
\newcommand{\mystrut}{\rule[-.5\baselineskip]{0pt}{2\baselineskip}}
% \newcommand{\mul}{\operatorname{mul}}
\newcommand{\even}{\operatorname{even}}
\newcommand{\sgn}{\operatorname{sgn}}
\newcommand{\iner}{\operatorname{iner}}
\newcommand{\rL}{\mathring{L}}

%%%COMMENT
\usepackage{soul}
\usepackage{cancel}
\newcommand{\rbf}[1]{\textbf{\color{red}#1}}

%%%THEOREM
\newtheorem{theorem}{Theorem}[section]
\newtheorem{lemma}[theorem]{Lemma}
\newtheorem{proposition}[theorem]{Proposition}
\newtheorem{corollary}[theorem]{Corollary}

\theoremstyle{definition}
\newtheorem{definition}[theorem]{Definition}
\newtheorem{observation}[theorem]{Observation}
\newtheorem{remark}[theorem]{Remark}
\newtheorem{example}[theorem]{Example}
\newtheorem{notation}[theorem]{Notation}
\newtheorem{question}[theorem]{Question}

% for title
\title{Bottleneck matrix}
\date{\vspace{-1cm}}
\begin{document}
\maketitle
\large

A \emph{rooted tree} is a tree $T$ with one of its vertex $v$ designated as the root.  Let $(T,v)$ be a rooted tree on $n$ vertices and $L$ its Laplacian matrix.  The \emph{Dirichlet operator} of $(T,v)$ is the matrix $\rL(T,v)$ obtained from $L$ by adding $1$ to the $v,v$-entry.  On the other hand, the \emph{bottleneck matrix} of $(T,v)$ is the matrix $B(T,v) = \begin{bmatrix} b_{ij} \end{bmatrix}$ such that $b_{ij}$ is the number of common vertices between the path from $i$ to $v$ and the path from $j$ to $v$.  It is known that $B(T,v)$ is the inverse of $\rL(T,v)$.  

Let $G$ be a graph and $L$ its Laplacian matrix.  Let $\alpha,\beta\subseteq V(G)$ be two subsets of vertices with $|\alpha| = |\beta| = k$ and $L(\alpha,\beta)$ the submatrix of $L$ obtained from $L$ by removing the rows in $\alpha$ and the columns in $\beta$.  The \textbf{all minors matrix tree theorem} states that $|\det(L(\alpha,\beta))|$ is the number of spanning forest that has exactly $k$ components and each component contains exactly one of the vertices in $\alpha$ and exactly one of the vertices in $\beta$.  (The sign of $\det(L(\alpha,\beta))$ can also be determined, but we do not really need it here.) 

Let $(T,v)$ be a rooted tree and $\rL$ its Dirichlet operator.  Let $T'$ be the tree obtained from $T$ by attaching a new vertex $u$ to $v$, and let $L$ be the Laplacian matrix of $T'$.  Thus, $L(\{u\},\{u\}) = \rL$ and has determinant $1$ by the matrix tree theorem.  Since $\rL$ is positive definite and all of its off-diagonal entries are negative or zero, $\rL$ is entrywisely positive. (why?)  Therefore, the $i,j$-th entry of $\rL^{-1}$ is 
\[\left|\frac{\det(\rL(\{j\},\{i\}))}{\det(\rL)}\right| = 
\left|\frac{\det(L(\{u,j\},\{u,i\}))}{1}\right|
,\]
which is the same as the bottleneck matrix of $(T,v)$.

\section*{Problems}
\begin{enumerate}
\setlength\itemsep{2em}
\item Pick a rooted tree on $4$ vertices.  Find its Dirichlet operator and its bottleneck matrix. 
\item Let $(P_4,v)$ be a rooted tree where $v$ is one of the endpoints of $P_4$.  Find its Dirichlet operator and its bottleneck matrix. 
\item Let $(K_{1,3},v)$ be a rooted tree where $v$ is the center vertex.  Find its Dirichlet operator and its bottleneck matrix. 
\item Use Rayleigh quotient to show that the Dirichlet operator of a rooted tree is always positive definite and invertible.
\item Prove that the inverse of the Dirichlet operator of a rooted tree is entrywisely positive.  [Hint: If $A$ is symmetric and its eigenvalues are in $(0,1)$, then $(I - A)^{-1} = I + A + A^2 + \cdots$.] 
\end{enumerate}

% \newpage
% \section*{Questions to ponder}
% \begin{enumerate}
% \item 
% \end{enumerate}

\end{document}
