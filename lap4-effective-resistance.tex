\documentclass{article}

%%%PACKAGE
\usepackage{amsmath,amssymb}
\usepackage{amsthm}
%% \usepackage[pdftex,bookmarks=true]{hyperref}
\usepackage{cite}
%% \usepackage{enumerate}
% \usepackage{url}
% \usepackage{hyperref}
% \usepackage{systeme}
%\usepackage{siunitx}
% \usepackage{multicol}
% \usepackage{systeme}

% for drawing graphs
\usepackage{tikz}
% \tikzset{every picture/.style={thick}}
\tikzset{every node/.style={draw, circle, inner sep = 2pt}}
\usetikzlibrary{arrows}

% for margins
\usepackage[margin=1in]{geometry}

% for font
\usepackage{euler}
\usepackage[OT1]{eulervm}
\renewcommand{\rmdefault}{pplx}

% \setlength{\parindent}{0pt}  %no indenting

% MACROS
\newcommand{\trans}{^\top}
\newcommand{\adj}{^{\rm adj}}
\newcommand{\cof}{^{\rm cof}}
\newcommand{\inp}[2]{\left\langle#1,#2\right\rangle}
\newcommand{\dunion}{\mathbin{\dot\cup}}
\newcommand{\bzero}{\mathbf{0}}
\newcommand{\bone}{\mathbf{1}}
\newcommand{\ba}{\mathbf{a}}
\newcommand{\bb}{\mathbf{b}}
\newcommand{\be}{\mathbf{e}}
\newcommand{\bp}{\mathbf{p}}
\newcommand{\bq}{\mathbf{q}}
\newcommand{\bx}{\mathbf{x}}
\newcommand{\by}{\mathbf{y}}
\newcommand{\bz}{\mathbf{z}}
\newcommand{\bu}{\mathbf{u}}
\newcommand{\bv}{\mathbf{v}}
\newcommand{\bw}{\mathbf{w}}
\newcommand{\tr}{\operatorname{tr}}
\newcommand{\nul}{\operatorname{null}}
\newcommand{\rank}{\operatorname{rank}}
%\newcommand{\ker}{\operatorname{ker}}
\newcommand{\range}{\operatorname{range}}
\newcommand{\Col}{\operatorname{Col}}
\newcommand{\Row}{\operatorname{Row}}
\newcommand{\spec}{\operatorname{spec}}
\newcommand{\vspan}{\operatorname{span}}
% \newenvironment{sol}{\medskip\noindent {\bf Solution.}}{\newpage}
\newcommand{\mystrut}{\rule[-.5\baselineskip]{0pt}{2\baselineskip}}
% \newcommand{\mul}{\operatorname{mul}}
\newcommand{\even}{\operatorname{even}}
\newcommand{\sgn}{\operatorname{sgn}}
\newcommand{\iner}{\operatorname{iner}}

%%%COMMENT
\usepackage{soul}
\usepackage{cancel}
\newcommand{\rbf}[1]{\textbf{\color{red}#1}}

%%%THEOREM
\newtheorem{theorem}{Theorem}[section]
\newtheorem{lemma}[theorem]{Lemma}
\newtheorem{proposition}[theorem]{Proposition}
\newtheorem{corollary}[theorem]{Corollary}

\theoremstyle{definition}
\newtheorem{definition}[theorem]{Definition}
\newtheorem{observation}[theorem]{Observation}
\newtheorem{remark}[theorem]{Remark}
\newtheorem{example}[theorem]{Example}
\newtheorem{notation}[theorem]{Notation}
\newtheorem{question}[theorem]{Question}

% for title
\title{Effective resistance}
\date{\vspace{-1cm}}
\begin{document}
\maketitle
\large

The \emph{Moore--Penrose pseudoinverse} of a symmetric matrix is the matrix $M$ that acts as the inverse of $M$ on $\Col(M)$ and has kernel $\ker(M)$, denoted by $M^\dagger$.  If $D$ is diagonal, then $D\dagger$ is obtained from $D$ by replacing each nonzero entry with each reciprocal.  For general symmetric matrix $M$ with diagonalization $M = QDQ\trans$, its pseudoinverse is $M^\dagger = QD^\dagger Q\trans$.

Let $G$ be a connected graph on $n$ vertices and $L$ its Laplacian matrix.  It is known that $L\bone = \bzero$ and $\rank(L) = n-1$, so $L$ is not invertible.  However, if we focus on $\bone\trans$, the space of vector orthogonal to $\bone$, then $L$ acts as an invertible function on $\bone\trans$.  Therefore, $L^\dagger$ sends a vector $\by\in\bone\trans$ to a unique vector $\bx\in\bone\trans$ and sends $\bone$ to $\bzero$.  

Let $G$ be a connected graph on $n$ vertices and $L$ its Laplacian matrix.  We may view $G$ as an electric circuit where each edge represents a wire of resistance $1$.  If the current flows into vertex $i$ and out from vertex $j$ with $1$ ampere and result in the voltage loss $V$, then we say the \emph{effective resistance} from $i$ to $j$ is $V/I$.  

Let $\be_1,\ldots,\be_n$ be the columns of $I_n$ and $\bv_{ij} = \be_i - \be_j$.  Then 
\[\bu_{ij}\trans L^\dagger\bu_{ij}\]
is the effective resistance from $i$ to $j$.  

In fact, the same idea works for weighted graph, where we set the weight of the edge $\{i,j\}$ as $\frac{1}{r}$ if the resistance of the wire from $i$ to $j$ is $r$ ohm.

\section*{Problems}
\begin{enumerate}
\setlength\itemsep{2em}
\item Let $J_3$ be the $3\times 3$ all-ones matrix.  Find $J_3^\dagger$.   
\item Let $L$ be the Laplacian matrix of $K_{2,2}$.  Find $L^\dagger$.
\item Let $L$ be the Laplacian matrix of $K_4$.  Find a vector $\bx\in\bone\trans$ such that 
\[L\bx = \begin{bmatrix} 1 \\ -1 \\ 0 \\ 0 \end{bmatrix}.\]
\item Let $G$ be a connected graph and $L$ its Laplacian matrix.  Define a function $f:\bone\trans\rightarrow\bone\trans$ by $f(\bx) = L\bx$.  Prove that $f$ is a bijective linear function.
\item Let $i$ and $j$ be two vertices of distance two on $C_4$.  Find the resistance distance from $i$ to $j$.
\item Find out what is the Moore--Penrose pseudoinverse for a general matrix.  Then find the pseudoinverse of $J_{3,4}$, the $3\times 4$ all-ones matrix.
\end{enumerate}

% \newpage
% \section*{Questions to ponder}
% \begin{enumerate}
% \item 
% \end{enumerate}

\end{document}
