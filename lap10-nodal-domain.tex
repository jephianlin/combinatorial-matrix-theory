\documentclass{article}

%%%PACKAGE
\usepackage{amsmath,amssymb}
\usepackage{amsthm}
%% \usepackage[pdftex,bookmarks=true]{hyperref}
\usepackage{cite}
%% \usepackage{enumerate}
% \usepackage{url}
% \usepackage{hyperref}
% \usepackage{systeme}
%\usepackage{siunitx}
% \usepackage{multicol}
% \usepackage{systeme}

% for drawing graphs
\usepackage{tikz}
% \tikzset{every picture/.style={thick}}
\tikzset{every node/.style={draw, circle, inner sep = 2pt}}
\usetikzlibrary{arrows}

% for margins
\usepackage[margin=1in]{geometry}

% for font
\usepackage{euler}
\usepackage[OT1]{eulervm}
\renewcommand{\rmdefault}{pplx}

% \setlength{\parindent}{0pt}  %no indenting

% MACROS
\newcommand{\trans}{^\top}
\newcommand{\adj}{^{\rm adj}}
\newcommand{\cof}{^{\rm cof}}
\newcommand{\inp}[2]{\left\langle#1,#2\right\rangle}
\newcommand{\dunion}{\mathbin{\dot\cup}}
\newcommand{\bzero}{\mathbf{0}}
\newcommand{\bone}{\mathbf{1}}
\newcommand{\ba}{\mathbf{a}}
\newcommand{\bb}{\mathbf{b}}
\newcommand{\be}{\mathbf{e}}
\newcommand{\bp}{\mathbf{p}}
\newcommand{\bq}{\mathbf{q}}
\newcommand{\bx}{\mathbf{x}}
\newcommand{\by}{\mathbf{y}}
\newcommand{\bz}{\mathbf{z}}
\newcommand{\bu}{\mathbf{u}}
\newcommand{\bv}{\mathbf{v}}
\newcommand{\bw}{\mathbf{w}}
\newcommand{\tr}{\operatorname{tr}}
\newcommand{\nul}{\operatorname{null}}
\newcommand{\rank}{\operatorname{rank}}
%\newcommand{\ker}{\operatorname{ker}}
\newcommand{\range}{\operatorname{range}}
\newcommand{\Col}{\operatorname{Col}}
\newcommand{\Row}{\operatorname{Row}}
\newcommand{\spec}{\operatorname{spec}}
\newcommand{\vspan}{\operatorname{span}}
% \newenvironment{sol}{\medskip\noindent {\bf Solution.}}{\newpage}
\newcommand{\mystrut}{\rule[-.5\baselineskip]{0pt}{2\baselineskip}}
% \newcommand{\mul}{\operatorname{mul}}
\newcommand{\even}{\operatorname{even}}
\newcommand{\sgn}{\operatorname{sgn}}
\newcommand{\iner}{\operatorname{iner}}
\newcommand{\rL}{\mathring{L}}

%%%COMMENT
\usepackage{soul}
\usepackage{cancel}
\newcommand{\rbf}[1]{\textbf{\color{red}#1}}

%%%THEOREM
\newtheorem{theorem}{Theorem}[section]
\newtheorem{lemma}[theorem]{Lemma}
\newtheorem{proposition}[theorem]{Proposition}
\newtheorem{corollary}[theorem]{Corollary}

\theoremstyle{definition}
\newtheorem{definition}[theorem]{Definition}
\newtheorem{observation}[theorem]{Observation}
\newtheorem{remark}[theorem]{Remark}
\newtheorem{example}[theorem]{Example}
\newtheorem{notation}[theorem]{Notation}
\newtheorem{question}[theorem]{Question}

% for title
\title{Nodal domain}
\date{\vspace{-1cm}}
\begin{document}
\maketitle
\large

The use of a Fiedler vector is not limited to trees.  

Let $G$ be a connected graph and $\bv = (v_1,\ldots,v_n)$ a Fiedler vector of $G$.  Again, we let 
\[\begin{aligned}
N_+(\bv) &= \{i \in V(G): v_i > 0\}, \\ 
N_-(\bv) &= \{i \in V(G): v_i < 0\}, \\
N_0(\bv) &= \{i \in V(G): v_i = 0\}.
\end{aligned}\]
It is known that $G[N_+(\bv)\dunion N_0(\bv)]$ and $G[N_-(\bv)\dunion N_0(\bv)]$ are connected.  Moreover, if the algebraic connectivity has multiplicity $1$, then $G[N_+(\bv)]$ and $G[N_-(\bv)]$ are also connected.

In general, let $G$ be a graph on $n$ vertices and $\bv\in\mathbb{R}^n$.  We may define $N_+(\bv)$, $N_-(\bv)$, and $N_0(\bv)$ as above.  A \emph{positive (negative) strong nodal domain} is a component of the induced subgraph $G[N_+(\bv)]$ ($G[N_-(\bv)]$).  A \emph{positive (negative) weak nodal domain} is a component of the induced subgraph $G[N_+(\bv)\dunion N_0(\bv)]$ ($G[N_-(\bv)\dunion N_0(\bv)]$).  
We may define $\mathfrak{S}(\bv)$ as the number of strong nodal domains of $\bv$ and $\mathfrak{W}(\bv)$ as the number of weak nodal domains of $\bv$.  

Let $G$ be a connected graph on $n$ vertices, $\lambda_1 \leq \cdots \leq \lambda_n$ its Laplacian eigenvalues, and $\bv_1,\ldots,\bv_n$ the corresponding eigenvectors.  By the Perron--Frobenius theorem, $\bv_1$ is either all positive or all negative, so $\mathfrak{S}(\bv_1) = \mathfrak{W}(\bv_1) = 1$.  From above, we know $\mathfrak{W}(\bv_2) = 2$, and, provided that the multiplicity of $\lambda_2$ is one, $\mathfrak{S}(\bv_2) = 2$.  This behavior can be extended.  The \textbf{discrete nodal domain theorem} states that 
\[\mathfrak{W}(\bv_k) \leq k \text{ and } \mathfrak{S}(\bv_k) \leq k + m - 1,\]
where $m$ is the multiplicity of $\lambda_k$.

More details on this topic can be found in \cite{BLS07}.

\begin{thebibliography}{9}
\bibitem{BLS07}
T.~Biyiko{\u g}u, J.~Leydold, and P.~F. Stadler.
\newblock {\em Laplacian {E}igenvectors of {G}raphs: {P}erron--{F}robenius and
  {F}aber--{K}rahn Type Theorems}.
\newblock Springer-Verlag Berlin Heidelberg, 2007.
\end{thebibliography}

\section*{Problems}
\begin{enumerate}
\setlength\itemsep{2em}
\item Pick a $\bv_2$ for $C_4$ and draw the strong and weak nodal domains of $G$ for $k = 2$.
\item Find a graph and a $k$ such that $\mathfrak{S}(\bv_k) > k$.  
\item Use software if necessary, pick a graph on $10$ or more vertices and then illustrate its nodal domains for some $k$.
\item Prove the discrete nodal domain theorem.  
\end{enumerate}

% \newpage
% \section*{Questions to ponder}
% \begin{enumerate}
% \item 
% \end{enumerate}

\end{document}
