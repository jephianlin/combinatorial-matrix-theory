\documentclass{article}

%%%PACKAGE
\usepackage{amsmath,amssymb}
\usepackage{amsthm}
%% \usepackage[pdftex,bookmarks=true]{hyperref}
\usepackage{cite}
%% \usepackage{enumerate}
% \usepackage{url}
% \usepackage{hyperref}
% \usepackage{systeme}
%\usepackage{siunitx}
% \usepackage{multicol}
% \usepackage{systeme}

% for drawing graphs
\usepackage{tikz}
% \tikzset{every picture/.style={thick}}
\tikzset{every node/.style={draw, circle, inner sep = 2pt}}
\usetikzlibrary{arrows}

% for margins
\usepackage[margin=1in]{geometry}

% for font
\usepackage{euler}
\usepackage[OT1]{eulervm}
\renewcommand{\rmdefault}{pplx}

% \setlength{\parindent}{0pt}  %no indenting

% MACROS
\newcommand{\trans}{^\top}
\newcommand{\adj}{^{\rm adj}}
\newcommand{\cof}{^{\rm cof}}
\newcommand{\inp}[2]{\left\langle#1,#2\right\rangle}
\newcommand{\dunion}{\mathbin{\dot\cup}}
\newcommand{\bzero}{\mathbf{0}}
\newcommand{\bone}{\mathbf{1}}
\newcommand{\ba}{\mathbf{a}}
\newcommand{\bb}{\mathbf{b}}
\newcommand{\be}{\mathbf{e}}
\newcommand{\bp}{\mathbf{p}}
\newcommand{\bq}{\mathbf{q}}
\newcommand{\bx}{\mathbf{x}}
\newcommand{\by}{\mathbf{y}}
\newcommand{\bz}{\mathbf{z}}
\newcommand{\bu}{\mathbf{u}}
\newcommand{\bv}{\mathbf{v}}
\newcommand{\bw}{\mathbf{w}}
\newcommand{\tr}{\operatorname{tr}}
\newcommand{\nul}{\operatorname{null}}
\newcommand{\rank}{\operatorname{rank}}
%\newcommand{\ker}{\operatorname{ker}}
\newcommand{\range}{\operatorname{range}}
\newcommand{\Col}{\operatorname{Col}}
\newcommand{\Row}{\operatorname{Row}}
\newcommand{\spec}{\operatorname{spec}}
\newcommand{\vspan}{\operatorname{span}}
% \newenvironment{sol}{\medskip\noindent {\bf Solution.}}{\newpage}
\newcommand{\mystrut}{\rule[-.5\baselineskip]{0pt}{2\baselineskip}}
% \newcommand{\mul}{\operatorname{mul}}
\newcommand{\even}{\operatorname{even}}
\newcommand{\sgn}{\operatorname{sgn}}
\newcommand{\iner}{\operatorname{iner}}
\newcommand{\rL}{\mathring{L}}

%%%COMMENT
\usepackage{soul}
\usepackage{cancel}
\newcommand{\rbf}[1]{\textbf{\color{red}#1}}

%%%THEOREM
\newtheorem{theorem}{Theorem}[section]
\newtheorem{lemma}[theorem]{Lemma}
\newtheorem{proposition}[theorem]{Proposition}
\newtheorem{corollary}[theorem]{Corollary}

\theoremstyle{definition}
\newtheorem{definition}[theorem]{Definition}
\newtheorem{observation}[theorem]{Observation}
\newtheorem{remark}[theorem]{Remark}
\newtheorem{example}[theorem]{Example}
\newtheorem{notation}[theorem]{Notation}
\newtheorem{question}[theorem]{Question}

% for title
\title{Characteristic set}
\date{\vspace{-1cm}}
\begin{document}
\maketitle
\large

Let $T$ be a tree and $v\in V(T)$.  Then each component of $T - v$ is called a \emph{branch} of $T$ at $v$.  Naturally, each branch has a unique vertex $u$ that is adjacent to $v$ in $T$, so we may also consider a branch as a rooted tree with root $u$.  

A \emph{weight} function on $T$ is a function $w$ that assigns a positive value to each branch of $T$ such that $w(F_1) > w(F_2)$ if $V(F_1) \supsetneq V(F_2)$ for any two branches $F_1$ and $F_2$.  Suppose $T$ is a tree equipped with a weight function $w$.  Then we may follow the steps below to find the ``central part'' of $T$:
\begin{enumerate}
\item Pick a vertex $v$ and observe the branches at $v$.  
    \begin{itemize}
    \item If there are two or more branches at $v$ that achieve the maximum weight among them, then return $\{v\}$ as the central part.
    \item If there is a unique branch $T_{u}(v)$ at $v$ that achieves the maximum weight among them, then let $v = u_i$ and repeat step 1.
    \end{itemize}
\item If step 1 ends up moving back and forth between two vertices $v$ and $u$, then return $\{v,u\}$ as the central part.
\end{enumerate}
Therefore, the ``central part'' can possibly be one vertex, or a pair of two adjacent vertices.  

One may draw the \emph{map ditree} as follows:  
\begin{enumerate}
\item Pick a vertex $v$ and observe the branches at $v$.  
    \begin{itemize}
    \item If there are two or more branches at $v$ that achieve the maximum weight among them, then draw a loop from $v$ to $v$.
    \item If there is a unique branch $T_{u}(v)$ at $v$ that achieves the maximum weight among them, then draw a directed edge from $v$ to $u$.  
    \end{itemize}    
\end{enumerate}
Based on the property of a weight function, each map ditree has either a unique loop at some vertex $v$ or a unique pair of vertices $v$ and $u$ with doubly directed edges in between.  Thus, either $\{v\}$ or $\{v,u\}$ is the central part.  

If $w(F)$ is the longest distance from the root of $F$ to a leaf of $F$, then the central part is called the \emph{center}.  If $w(F)$ is the number of vertices of $F$, then the central part is called the \emph{centroid}.  If $w(F)$ is the spectral radius of the bottleneck matrix of $F$, then the central part is called the \emph{characteristic set}.

\section*{Problems}
\begin{enumerate}
\setlength\itemsep{2em}
\item Pick a tree and let $w(F)$ be the longest distance from the root of $F$ to a leaf of $F$.  Draw the map ditree and find the center.
\item Pick a tree and let $w(F)$ be the number of vertices of $F$.  Draw the map ditree and find the centroid.
\item Pick a tree and let $w(F)$ be the spectral radius of the bottleneck matrix of $F$.  Draw the map ditree and find the characteristic set.
\item Prove that a map ditree must have either a loop or a pair of vertices $v$ and $u$ with doubly directed edges in between.  (Think about why only one of the two structures can happen and why it never happens at two places.)
\end{enumerate}

% \newpage
% \section*{Questions to ponder}
% \begin{enumerate}
% \item 
% \end{enumerate}

\end{document}
